%file included in thesis.tex


\chapter{Sensors}


\section{2D Sensors}

\subsection{Laser Range Finder}





\section{3D Sensors}


\subsection{Stereo Camera}



\subsubsection{Camera Calibration}
To work with stereo images the captured images need to be rectified, i.e. the images need
to be corrected for distortion introduced by the camera lenses. This is a part of
calibrating the camera, meaning that the intrinsic parameters of the camera are
determined. That is distortion coefficients, and other camera parameters. 

Figure \ref{cahp3:fig-lensdist} below shows how the camera lenses of the Stereo camera distorts the pictures
towards the edges of the lens. 

\begin{figure}[htbp]
    \centering
    \includegraphics[width=0.8\textwidht]{pics/left-rigth-camera-dist}
    \caption{The left and right camera distortion due to lens nonlinearities}
    \label{chap3:fig-lensdist}
\end{figure}



\subsection{Time-of-Flight Camera}




\section{Comparisons Between the Proposed Sensors}

\begin{table}[htbp]
    \centering
    \begin{tabular}{|c|c|c|c|c|c|c|c|c|}
        \hline
        Sensor & Range & Accuracy & FoV & Ang Res & Weight & Scan Freq & Power Cons &  Cost \\
        \hline
        LMS-100 & 20 m & 12 mm &  $270^{\circ}$ & $0.25-0.5^{\circ}$  & 1.1 kg    & 50 Hz & Not specified  & \$5500 \\
        \hline
        LMS-200 & 80 m & 30 mm &  $180^{\circ}$  & $0.25-1^{\circ}$  & 4.5 kg    & 75 Hz & Not specified &  \$5000 \\
        \hline
        UTM-30LX & 30 m & 30 mm & $270^{\circ}$ & $0.25^{\circ}$  & 210 g     & 40 Hz  &$<8$ W   &  \$5000 \\
        \hline
        URG-04LX & 4 m & 10 mm & $240^{\circ}$ & $0.36^{\circ}$ & 160 g  & 10 Hz & ca 2.5 W &  \$2400 \\
        \hline
        URG-04LX-URG01 & 5.6 m & 30 mm & $240^{\circ}$ & $0.36^{\circ}$ & 160 g & 10 Hz & ca 2.5 W & \$1100 \\
        \hline
        SR3000-ToF Camera & 7.5 m & 30 mm & $47.5$ x $ 39.6 ^\circ$ & 176 x 144 & -  & 25
        fps & 18 W max & \$10000 \\
        \hline
    \end{tabular}
    \caption{Comparison between the proposed sensors}
    \label{tab-chap3-sensors}
\end{table}
***FIX TABLE \ref{tab-chap3-sensors}********


\section{Sensor Configurations}
There are a couple of possible sensor configurations available for this project. ***SHOULD
MAYBE BE IN THE IMPLEMENTATION CHAPTER?****

The Laser Range Finder should alway be used in every sensor configuration. This should be
the primary source of length measurements since this is the most accurate of all the range
devices. 

The stereo camera might be looked upon as a cheap type of the Time-of-Flight camera. So
the two possible configurations are:
\begin{itemize}
    \item Laser Range Finder and Stereo Camera
    \item Laser Range Finder and Time-of-Flight Camera
\end{itemize}
The Stereo Camera will give sparse range data, while the Time-of-Flight camera will give
dense range data. Assuming that t




