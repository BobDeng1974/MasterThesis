%this file is included in thesis.tex

\chapter{Introduction}
Robot navigation is a topic which have been researched heavily the last 4 decades. The
interest of making a machine which can navigate its surroundings and reason what to do
next have always been a dream for many researchers. 

Making a robot which can do dull, dirty  or dangerous work have always been seen on as a
good use of robots. This because the do not tier as a human would do performing the same
task over and over again. A robot is expendable, a human is not, allowing it to go into
dangerous places and investigate. Imagine a robot go into a disaster area and search for
survivors and map the area in advance for the rescue team. The third of the ``Ds'' are the
dirty aspect, which can both be dull and dangerous, and a human might not be able to go
into such an area. 

The Pipe Inspection Konda (PiKo) is a seven-segment robot with active wheels developed by
SINTEF Applied Cybernetics. This is one of the numerous generations of snakelike robots
developed by SINTEF, starting with Anna Konda, which originally was designed to be a fire
hose and assist firemen in areas where explosion danger are imminent, or the smoke is too
thick for the fire men to enter, like tunnel fires and so on. 

Some of PiKos joints have 2 degrees of freedom, which makes it able to climb vertical ducts. This
because it can span itself as an S, and the friction from the wheels keeps it in place.
This makes it a unique tool for pipe and duct inspection, in full 3D movement. 
For this to work properly there are a number of obstacles which have to be overcome. 

First, theres the navigation in the duct or pipe network. How to keep track of where the robot
are, and even more important, where it has been. This is a difficult problem, since there
are no absolute navigation system available to the robot, i.e. GPS or other beacon based
navigation, available to the robot. This means that the robot must rely on some kind of
dead reckoning navigation. This means that the position is integrated from acceleration or
speed measurements, which gives a very more and more uncertain position estimate as time
goes by. This means as the mission time progresses the uncertainty of the map increases,
which might give very erroneous results at the end of the mission.

Second, which is derived form the first obstacle, is that the robot need to tag pipe
defects and other anomalies in the pipeline. This anomalies should be marked as accurate
as possible. This demands a great deal from the navigation system with regard to accuracy. 

Third, since this is a robot, which are designed to be autonomous and optimized for long
missions, the computational abilities and resources in the robot are limited. This calls
for that the navigation and mapping step should not be too computationally intensive.
Especially the mapping of areas in 3D might be very computationally intensive, and demand
much storage. This calls for a sparse and representation of the environment, even though
the computer capacity are increasing with the years, the robot are served with having as
sparse representation as possible. 

The problem with sparse representation is that it requires much more reasoning from the 
robot. This means that the robot has to interpret the environment into more abstract form,
i.e. it has to recognize a junction as a Y-junction and a turn as a L-bend. This demand a
great deal from the control system and the sensors. 

The sensors of the robot are important. This represents the way the robot senses the world
and influences a great deal on how it will map it's surroundings. 


\section{Motivation}


\section{Assumptions and Criteria}


\section{Structure of this report}
This report is divided into 9 chapters, whereas Chapter 1 is the introduction, Chapter 2
treats the mathematical background of the measurements principles involved. It describes
the sensors, different map representations and sensor fusion schemes. Last theres a
mapping on similar applications around the world.

Chapter 3 describes the sensors which are used in this project, and discusses the pro's
and con's of the sensors. Chapter 4 describes the sensor fusion approach chosen for the
project while Chapter 5 treats the map representation. 

The implemented solution are described in detail in Chapter 6, while the test setup are
shown in Chapter 7. The results are discussed in Chapter 8, and conclusions are drawn in
Chapter 9. Future work and solutions are also plotted in Chapter 9.

TODO: fiks labels og referanser. 

