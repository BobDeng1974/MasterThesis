
\chapter{Conclusions}
\label{chap9}
This report have proposed a partial navigation system for a pipe inspection robot using three
types of sensors, namely a Laser Range Finder, a Time-of-Flight camera and a stereo
camera. The system uses a modular three-layer approach to the implementation, where the lower layer is the
sensor layer, which transforms the sensor information into a common coordinate
representation. The middle layer handles the interpretation of the sensor data, which include cylinder fit-
and line fit algorithms. The third layer handles the world representation, that keeps track of
the current explored areas. Further, pipe profile matching, path planning algorithms and command
algorithms might be implemented in the third layer, but this is a topic for future work. 

The different sensors have been investigated and their abilities have been
evaluated, and possible difficulties by using the proposed sensors have been evaluated.
This shows that the laser range finder provides reliable results in the plane without much
filtering and treatment of the sensor data. But does only provide measurements in the
plane, which means that it easily misses obstacles that are close to the ground and does
not cross the measurement plane of the 2D sensor. 

The time-of-flight camera on the other hand,
need more preparation before it can be utilized to create a map of the surroundings,
mostly because it provides more denser information, and this information is more prone to
noise. The intrinsic parameters of the camera where determined, and corresponded good to
the values given by the manufacturer. No calibration for the range data were preformed.
Mostly because it where thought to not impact the system performance significantly. 

The stereo cameras intrinsic parameters and lens distortion coefficients where estimated.
As expected this showed severe distortions and offset of the principal axes which impacted
the field-of-view of the stereo rig. The cameras did not preform that well in the given
environments, this is mostly due to the quality of the camera, and the lack of synthetic
lighting in the scene. Some enhancement of the images might have given better matching
results. But this were not tested in the report. 

A proposed representation scheme of the sensor data were also implemented. This is a
topological representation of the world, which is based on much reasoning and
interpretation of the sensor data. A set of nodes are defined, this can be pipeline
junctions, like a pipeline bend, or a T-junction and likewise. It can also be any other
feature that is not like the usual featureless straight pipeline. As said, this relies
greatly on the interpretation of the sensors, which prove difficult with the proposed
algorithms. Although, no matching algorithms from the sensor data to the world
representation is proposed, the topological approach for the world representation were
chosen because of the simple and demanding less memory than other mapping approaches, like
the Occupancy Grid, which demand much memory, especially if implemented in three
dimensions. 

The implemented sensor interpretation algorithms did not work according to the planned
results. The algorithms are least-squares based algorithms, and does not work well when
now selection of the data set is preformed. The topic of segmenting a data set into
different regions are a difficult but crucial topic when the interpreting sensor data from
complex scenes. Unfortunately, time did not allow for developing a good solution to this
problem. One possible solution is to use another algorithm for surface fit, like the
RANSAC algorithm, which is widely used in computer graphics applications. This approach
employs an internal selection of points that it thinks belongs to the data set.
\cite{ransac}. The two dimensional line fit from the Laser Range Finder worked adequately,
but suffered from the same problems as the three dimensional case did. 



\section{Future Work}
It is numerous of things that need to be looked into further. This section will try to 
summarize this points. 

\paragraph{Obstacle avoidance and Path Planning} For the 

\paragraph{Local navigation}

\paragraph{Segmenting of range images}

\paragraph{Fusion of Time-of-Flight and Stereo Camera}

\paragraph{Profile Matching}

\paragraph{Matching for global positioning}

\paragraph{Robust Reasoning}



