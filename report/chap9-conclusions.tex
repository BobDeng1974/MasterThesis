
\chapter{Conclusions}
\label{chap9}
To draw conclusions from the results in Chapter \ref{chap7} and \ref{chap8} are difficult.
The proposed algorithms lacked selection and segmentation of the data. This was fatal
because the system needed to reason and the background of what it was looking for was
wrong. This shows the importance of selecting the right conditions and terms before
putting the data into the algorithms.

The representation is good, although it was not tested properly in this thesis, it have
the strengths that graph representations have, search and path planning. Path planning is
really just a shortest-path problem which can be implemented relatively easy in a fast and
efficient way. This also allows for easy interpretation back to the user, which then can
investigate the nodes which have been marked with anomalies. The backside of this is that
it requires much reasoning from the robot, which demand robust recognition algorithms.


This project can be seen as a mapping project on what to be on the look-out for when
designing this kind of system. This include, sensor calibration, sensor data
segmentation



\section{Further Work}


