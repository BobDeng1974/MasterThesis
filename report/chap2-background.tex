%included in thesis.tex,


\chapter{Background}
This chapter will try to outline the basics needed to understand this report. Mathematical
knowledge are crucial to understand the principles used in robot navigation and decision
making. 


\section{Mathematical description and modelling of a Snake robot}


\section{State-of-the-Art Literature study}


\subsection{2D Sensors}

\subsubsection{Laser Range Finders}


When looking at Laser Range Finder, there are 3 techniques for determining the distance,
\emph{optical triangulation}, \emph{pulse Time-of-Flight} and \emph{frequency modulated
continuous wave} (FMCW). \cite{laser-ranging-critical-review}

\paragraph{Optical Triangulation}
Optical triangulation is performed by using a light source and a placed at known distance
from each other. This distance is called the baseline. This is a triangulation setup which
allows for the calculation of the intersection point between the lines. 

\begin{figure}[htbp]
    \caption{Optical Triangulation setup}
    \label{fig:optical-triangulation}
\end{figure}
***Lag figur*** TRENGER JEG DETTE?*****

This method has some fundamental limits. \cite{laser-ranging-critical-review} 

This technique is mostly used when the measured distance is short and the need for
extremely precise measurements. For example digitalization of 3D objects.


\paragraph{Pulsed Time-of-Flight}
The laser light which is emitted are amplitude modulated, and the received signal is
compared to the emitted signal and the phase delay is measured. This is then directly
proportional to the distance the light traveled. 


\paragraph{Frequency Modulated Continuous Wave}


\subsubsection{Sonar}


\subsubsection{Infrared Sensors and Proximity Sensors}


\subsection{3D Sensors}


\subsubsection{Time-of-Flight Cameras}
This is the article which I'm going to cite. \cite{sr3000}



\subsubsection{Stereo Vision}
Stereo vision is an inexpensive way of finding range measurements. The setup is two or
more cameras some distance away from each other. This distance is called the
\emph{baseline}. The camera images are then compared and a common reference point is
found. The Difference between the images are then used together with the baseline distance
to triangulate the common reference point and then find the distance to the observed
object. 



\subsection{Map Data Representation}
The representation of sensor data is important for the functionality of the robot. This is
dependant on the amount of processing power and capacity that is available at the robot.
The area which the robot is to map is also of great importance when choosing the
representation. There are a number of representations that are tried out, and each one has its good and
bad abilities. 

The major representation methods which is used in literature are summarized in the below
sections. There are two major ideas when it comes to map representation, and that is
metric and topological.

Metric representation uses the sensor data as we see it. It is the direct Cartesian
representation of the way the world is. Examples of this are CAD drawings of floor plans
and housing. This maps are often large and inexact, because of many kinds of
uncertainties.

Topological representations uses a more abstract way. The world are represented by graphs,
which is nodes and links between them. This are a sparse and efficient way of representing
the world, but need more processing when the map is built. This kind of maps are not
exact, because they are not expected to be. They simply describe the connection between
different kind of objects, and all objects might have attributes describing how it really
looks. This method is useful in highly structured environments, such as pipelines, sewers,
and office landscapes. This because a series of junctions might tell you where you are
than just the metric information, which might be quite erroneous. 


\subsubsection{Occupancy Grid Maps}
Occupancy grid maps are a metric approach to the mapping problem and are widely used in 
robotic mapping, mostly because of it is simple to implement and use. It was developed 
by Elfes and Moravec in the mid 1980s, \cite{elfes}, \cite{moravec}. The method is quite 
robust and it is simple to implement with many kinds of sensors. This method assumes that 
the robots pose is known.

This method divides the world into grids with probabilities that the grid is occupied. It
starts with all grids at 50 \%, equal probability that the space is occupied or not. As
the robot moves around it updates the grids according to its sensor readings. When for
example employing laser range finders, the grid at the distance reported from the range
finder are marked as occupied according to some uncertainty set by the accuracy of the
range finder. The grids between the possible detected obstacle will then be decreased
because the probability of obstacles are less. 

Occupancy maps are not the most computationally effective way to represent the world,
especially when it is big. It is cumbersome and may lead to problems when dealing with
cyclic environments. This because of the uncertainty in the sensor measurements and robots
pose and odometry. 



\subsubsection{Topological Maps}
Object maps are topological maps and represents the sensed world in the form of predefined nodes and links
between them. Each node has a set of attributes, like length, links to doors etc. This is
a compact way to express the world and is much more computationally effective with larger
maps than the occupancy grid maps might be. Examples of such nodes are corridors,
junctions and dead ends, but this nodes must be suited to the application of the robot. 

The problem with this representation is that it needs a lot of input \'a priori. The
operator creates and inputs the map to the robot, which uses this map for navigation. The
largest challenge with this is to make the robot aware of where it is. It needs in some
way to recognize its surroundings, and match it to the \'a priori map.


\subsubsection{Mixed Approaches}
This is when you take the best of both worlds. The easy and effective way of the metric
maps, and mix them with the abstract topological maps. This will help reduce the size of
the maps in the robots memory, and also make it more computationally effective because the
maps are sparse. 

******?????The mixed approaches are usually not computed on-line***???? (CHECK THIS). This utilizes
the metric approach for the rooms that have a lot of details, chairs, desks etc. and for
corridors and less detailed areas the topological approach, when the metric and
geographical information are not that important. 

This requires lots of processing of the sensor and map data, and should be done off-line
after the robot's mission is finished. 


\subsection{Sensor Fusion Techniques}
Cite the article \cite{sensor-fusion-mobile-robots}. Write about the different kind of
fusion techniques. 

\begin{itemize}
    \item Low-level Fusion with unknown statistics
        \begin{itemize}
            \item Rule-based
            \item Geometric and topological maps
        \end{itemize}
    \item Low-level Fusion with known statistics in centralized approaches
        \begin{itemize}
            \item Kalman Filter and probabilistic approaches
        \end{itemize}
    \item Low-level fusion with known statistics in decentralized architectures
        \begin{itemize}
            \item 
        \end{itemize}
    \item High-level Fusion
        \begin{itemize}
            \item Behaviour-based architectures
        \end{itemize}
\end{itemize}


\subsubsection{Low-level Fusion}
Low-level Fusion are methods that uses mathematical principles as the Kalman Filter or
Bayesian fusion. 


\subsubsection{High-level Fusion}



\subsubsection{Interpolation Techniques}



\subsubsection{Probabilistic Techniques}


\subsection{Feature Extraction}
The need for feature extraction should be obvious when dealing with autonomous robot
navigation. For the robot to know where it is it needs to recognize its surroundings. This
is a difficult step, because computers are not known to be very good at this, as the human
brain is. 

By feature extraction it is meant to recognize one point or feature in one instant and
find the same point at the next time instant. This can be imagined difficult in some
surroundings where theres little individual detail. 

This problem is relaxed a bit when the surroundings are confined to pipelines. The problem
here is that the interior of the pipeline are not rich in detail, and is mostly the same
everywhere the robot travels, except in the different kinds of junctions. By feature
extraction in this report the meaning is to recognize the different types of junctions
that the robot will travel through. 

\cite{theilemann-breivik}


\subsection{Real World Applications}

Remember MAKRO! The German snake robot used for sewer inspections. 



