%file included in thesis.tex


\chapter{Data Representation}
This section will describe the world representation used in the implementation.

\section{World Representation}
The world representation should have a number of qualities. First it must represent the
world in a satisfactory way. It should be easy to update and take up as little memory in
the robots memory as possible. Third it should be easy to understand for a human operator
without too much post-processing. 

When choosing how to represent the world one must take into account how much reasoning and
processing the robot should do with the sensor data. The more abstract world
representation, the more reasoning and processing must be done at the robot. On the
upside, abstract representations takes up less space in memory, and can be more
computationally effective when it comes to path finding and planning, than the less
abstract map representations. 

The problem when a robot has to reason, is that it might take the wrong decisions, and the
risk of erroneous map making are high. 

\subsection{Nodes}
The representation in this project is a kind of topological map. It is abstract in the way
that the world are classified into given nodes. Known features of the world are recognized
and represented as nodes in a graph. 

This means that the world need to be classified into predefined nodes before the mission
is started. In this project the nodes are junctions, or bends that are easily recognized
because of their fundamental difference to the straight pipe segments. 

The nodes should have a number of attributes describing how the node is different from the
other nodes. These attributes are:
\begin{itemize}
    \item Node type
    \item Number
    \item Time stamp when discovered
    \item Previous Node
    \item Distance since last node
    \item Orientation
    \item Number of edges and the angles of those
    \item List of anomalies
\end{itemize}

Most of this attributes are self-explaining, but the last attribute, \emph{list of
anomalies} deserves a closer explanation. 

Anomalies are those things that the robot should look after, and record. An anomaly are
everything that is not expected. In a pipe environment the surroundings are assumed to be
in some way, because they are designed that way. Everything that is not as expected should
be recorded for later inspections. 

This list will contain every anomaly from the previous node to the next one, where the
list is stored. The list will contain a time stamp of when it was discovered, distance
traveled from the previous node, and a remark why it was detected as an anomaly. 

Since the anomalies will be detected before the node which will contain the list are
detected, the list is temporary stored in the robots memory until the next node are
detected. 


\section{Map-building From the Sensor Data}


\subsection{Cylinder or Cone Fit from 3D Sensor Data}
The cylinder fit is a Least-Squares Gauss-Newton algorithm. This algorithm are very
useful since it does not require the calculation of the Second order derivatives, or the
Hessian matrix. This calculations are very cumbersome and computationally expensive. 

The point cloud which is the output from the Time-of-Flight camera are divided into
sections with regard to the z-axis, i.e. the depth into the pipe. This intervals of 3D
data are fed into the algorithm and produces a set of cylinders which are fitted to the
pipe. The distance of the points from the proposed cylinder are output from the algorithm
and are used for determining if the cylinder fit is a good fit. 

\begin{figure}[htbp]
    \centering
   % \includegraphics[width=0.7\textwidth]{pics/cylinderfit}
    \caption{The set of cylinders fitted to the point cloud}
    \label{chap5:fig-cylinderfit}
\end{figure}
An example of a set of fitted cylinders are shown in Figure \ref{chap5:fig-cylinderfit}.
One might see that the axis are off. This is because noise and the least squares nature of
the fit. The noisy points will draw the cylinder off axis compared to the real one. 


\subsection{Line Fit and Arc Fit in 2D Sensor Data}
Line fit algorithm is a least-squares line fit algorithm. For this to work properly, a
selection of points assumed belonging to the line need to be fed into the algorithm. To
select this points a histogram is formed for both x- and y-direction of the points. This
will detect points which lie in a horizontal- and vertical lines. This points are selected
and fed into the line fit algorithm. 

This method involves a number of parameters and threshold values which need to be set and
tuned by the user. 



\section{Positioning in the World Representation}




