




\chapter{Matlab m-files}

\lstloadlanguages{Matlab}

\lstset{% general command to set parameter(s)
    language=Matlab,
    basicstyle=\scriptsize, % print whole listing small
    stringstyle=\ttfamily, % typewriter type for strings
    showstringspaces=false} % no special string spaces


\section{World representation}

\subsection{\emph{world.m}}
\begin{lstlisting}

%% World class, containing all the nodes. 
% This is the container class for all the nodes stored in the world
% 

classdef world
    properties %(SetAccess = private) % ?
    %% private attributes?
        nodes; % the nodes that are discovered. 
    end
    
    properties
    %% public attributes?
    pipe_diameter; % The diameter of the pipeworld
    
    end
    
    methods
 
        %%  Constuctor
        function [object] = world(diameter)
            
            object.nodes = []; % create an empty list.
            
            if nargin == 1
                start = start_node(0, diameter); % create a start node
                object.pipe_diameter = diameter;
            else
                start = start_node(0, 10);
                object.pipe_diameter = 10;
            end
            
%             object.nodes = [start];
%             object.nodes(1) = object.nodes(1).draw_at_position([0;0]);
%                                     
            object = object.addNode(start, [0; 0]);
        end
        
        %% Othre functions
        % add node function for adding nodes to the world
        % 
        
        function [object] = addNode(object, node, position, distance, orientation)        
            % When adding nodes, they need to be of the same type. Then it
            % should be easiest to convert them to the super class, node to
            % fit all the nodes into the same array.
            if nargin < 3
                disp('Too few arguments');
            elseif nargin == 3
                if ~strcmp(class(node), 'super_node')
                    % convert the node to super node and put into array
                    
                    node = node.super_node(); %convert the node to the super class node
                    node = node.setPosition(position);
                    
                    object.nodes = [object.nodes; node]; % add the node to the list
                    %object.nodes(end) = object.nodes(end).draw_at_position(position);
                else
                    % is of type node.
                    node = node.setPosition(position);
                    object.nodes = [object.nodes; node]; % add the node to the list
                    %object.nodes(end) = object.nodes(end).draw_at_position(position);
                end
                
            elseif nargin == 5 %% supplied the distance and orientation instead
                if ~strcmp(class(node), 'super_node')
                    % convert the node to super node and put into array
                    node = node.super_node(); %convert the node to the super class node
                    
                                    
                else
                    % is of type node.
%                     node = node.setPosition(position);
                    object.nodes = [object.nodes; node]; % add the node to the list
                    %object.nodes(end) = object.nodes(end).draw_at_position(position);
                end 
            end
        end
              
                
        function [] = draw(object)
            
            for i = 1:length(object.nodes)
                object.nodes(i) = object.nodes(i).draw_at_position(object.nodes(i).position);
                object.nodes(i).draw_edges(); % indicate what kind of junction it is.
            end
        end
        
        
        function [] = draw_edges(object)
            % algorithm: 1. find which nodes is connected to who.
            %            2. draw the edges between them, with according
            %            distances
            %            3. Draw the edges which are not connected to any
            %            4. finished
            
            % extra: Need to handle faults and errors, check if the
            % distance is correct. Check if there really is an edge at the
            % exiting at the bearing allowed by the node in question.
            
            for i = 1:length(object.nodes) % Assumes that the nodes array are sorted 
           
                if i ~= length(object.nodes)
                    distance = object.nodes(i+1).dist_prev_node;
                    
                    % draw the edges
                    
                    hold on
                    plot([object.nodes(i).position(1), object.nodes(i+1).position(1)],...
                        [object.nodes(i).position(2), object.nodes(i+1).position(2)]);
                    hold off;
                                        
                else
                    distance = 0; %no noeds after this one.
                    
                    disp('Last node in list');
                end
                
                
                
            end
            
            
        end
        
        
    end
end

\end{lstlisting}


\subsection{\emph{super\_node.m}}

\begin{lstlisting}
%% Class definition for the genreal node in the world representation. 
% The plan is to make this class a super class and let the different types
% of nodes inherit form this one.
% Description of the properties:
%           type: the type of the node.
%           number: is the number in the node, usually describing the
%                   number in which it is discovered
%           prev_node: is the link to the previous visited node.
%           discovered: is the timestamp when the node was recognized
%           diameter: the estimated diameter of the pipe
%           dist_prev_node: is the estimated distance to the previous node.
%           anomalies: A list of anomalies discovered between this node and
%                      the previous one,
%           orientation: The orientation of the node, relative to NED
% List of methods:
%           
classdef super_node
    properties
        type;
        number;
        orientation;
        prev_node;
        discovered;
        diameter;
        dist_prev_node;
        anomalies;
        color;
%     end
    
%     properties(GetAccess = protected, SetAccess = protected) % Drawing properties
        number_of_edges; % The number of edges connecting to the node
        angles_of_edges; % array with the angles of the connected edges. Relative to NED
        position;
    end
    
    
    methods
        
        %% constructor
        function [object] = super_node(number, orientation, prev_node, diameter, dist_prev_node)
            if nargin == 0 
                % do nothing, because the felds are empty.
            elseif nargin == 1 % maybe just a struct?
                if isstruct(number)
                    object.number = number.number;
                    object.orientation = number.orientation;
                    object.prev_node = number.prev_node;
                    object.diameter = number.diameter;
                    object.discovered = number.discovered;
                    object.dist_prev_node = number.dist_prev_node;
                    object.type = number.type;
                    object.anomalies = number.anomalies;
                    object.color = number.color;
                    object.number_of_edges = number.number_of_edges;
                    object.position = number.position;
                    object.angles_of_edges = number.angles_of_edges;
                end
                    
            else
                if ~isempty(number)
                    object.number = number;
                else
                    object.number = [];
                end
                if ~isempty(orientation)
                    object.orientation = orientation;
                else
                    object.orientation = [];
                end
                if ~isempty(diameter)
                    object.diameter = diameter;
                else
                    object.diameter = [];
                end
                if ~isempty(prev_node)
                    object.prev_node = prev_node;
                else
                    object.prev_node = [];
                end
                if ~isempty(dist_prev_node)
                    object.dist_prev_node = dist_prev_node;
                else
                    object.dist_prev_node = [];
                end
            end 
            if isempty(object.discovered)
                object.discovered = datevec(datestr(now, 0));
            end
        end
        
        
        
        %% Adding anomalies which have been observed during the travel
        
        % TODO need some error check to see that the anomaly is put into the 
        % table correctly.
        function [object] = addAnomaly(object, anomaly)
            object.anomalies = [object.anomalies; anomaly];
        end
        
        %% Set functions
 
        function [object] = setPostition(object, pos)
            
            if nargin == 1
                object.position = pos;
            else
                disp('Too few arguments');
            end
            
        end
       
        %% display functions
        
        function [object] = draw_at_position(object, pos) % Draws the node at pos x, y using fill
            or = (object.orientation);
            
            r = [cosd(or), -sind(or);
                 sind(or), cosd(or)];
        
            % Need to check that it is a column vector
            if size(pos) == 2 
                posb = r'*pos'; % transform to body coords
            else
                posb = r'*pos;
            end
            
            
            % need to rotate the box with regard to the orientation
            % transform to global coords
            pos1 = r*(posb + [-1; -1]);
            pos2 = r*(posb + [-1; 1]);
            pos3 = r*(posb + [1; 1]) ;
            pos4 = r*(posb + [1; -1]);
            
            x = [pos1(1), pos2(1), pos3(1), pos4(1)];
            y = [pos1(2), pos2(2), pos3(2), pos4(2)];
            
            hold on; % 
            fill(x, y, object.color);
            hold off;
            
            object.position = pos;
        end
        
        function [] = draw_edges(object)
            r = 1; %set defualt lenght of edges
            
            for i = 1:object.number_of_edges
                angle = (object.angles_of_edges(i) + object.orientation);
                
                x = r*cosd(angle) + object.position(1);
                y = r*sind(angle) + object.position(2);
             
                hold on;
                plot([object.position(1), x], [object.position(2), y], 'white', 'LineWidth', 3);
                hold off;
            end
        
        end
        
        
        
        function [object] = setPosition(object, position)
                    object.position = position;
        end
        
        function [object] = setOrientattion(object, orient)
                object.orientation = orient;
            
        end
        
        
    end 
end
 

\end{lstlisting}


\subsection{t\_junction.m}

\begin{lstlisting}
%% Class defs of T-junction
 
classdef t_junction < super_node
    properties    
    end
    methods
        
        function [t] = t_junction(number, orientation, prev_node, diameter, dist_prev_node)
            
            %% Default constuctor
            if nargin == 0
                args = {};
            else
                if ~isempty(number)
                    args{1} = number;
                end
                if ~isempty(orientation)
                    args{2} = orientation;
                end
                if ~isempty(prev_node)
                    args{3} = prev_node;
                end
                if ~isempty(diameter)
                    args{4} = diameter;
                end
                if ~isempty(dist_prev_node)
                    args{5} = dist_prev_node;
                end
            end
            
            %% calling super constructor.
            t = t@super_node(args{:});
            
            %% class specific
            
            t.color = 'green';
            t.type = 'T Junction';
            
            t.number_of_edges = 3;
            t.angles_of_edges = [0, 180, 270]; %standard  
        end
        
        function [lhs] = super_node(rhs)
            %% converting function
            lhs.number = rhs.number;
            lhs.type = rhs.type;
            lhs.orientation = rhs.orientation;
            lhs.prev_node = rhs.prev_node;
            lhs.discovered = rhs.discovered;
            lhs.diameter = rhs.diameter;
            lhs.dist_prev_node = rhs.dist_prev_node;
            lhs.anomalies = rhs.anomalies;
            lhs.color = rhs.color;
            lhs.number_of_edges = rhs.number_of_edges;
            lhs.angles_of_edges = rhs.angles_of_edges;
            lhs.position = rhs.position;
            
            lhs = super_node(lhs);
            
        end
        
        function [t] = setType(obj, nodetype)
            obj.type = nodetype;
            t = obj;
        end
    end
end

\end{lstlisting}

\subsection{start\_node.m}

\begin{lstlisting}
%% Class defs for the start node
 
classdef start_node < super_node
    properties
    end
    methods
        
        function [t] = start_node(orientation, diameter)
            
            %% Default constuctor
            if nargin == 0
                args = {};
            else
                args{1} = 1;
                args{3} = [];
                args{5} = [];
                if ~isempty(orientation)
                    args{2} = orientation;
                else 
                    args{2} = [];
                end
                if ~isempty(diameter)
                    args{4} = diameter;
                else
                    args{4} = [];
                end
                
            end
            
            %% calling super constructor.
            t = t@super_node(args{:});
            
            %% class specific
            
            t.color = 'yellow';
            t.type = 'Start Node';
            
            t.number_of_edges = 1;
            t.angles_of_edges = 180; %standard  
        end
        
        function [lhs] = super_node(rhs)
            %% converting function
            lhs.number = rhs.number;
            lhs.type = rhs.type;
            lhs.orientation = rhs.orientation;
            lhs.prev_node = rhs.prev_node;
            lhs.discovered = rhs.discovered;
            lhs.diameter = rhs.diameter;
            lhs.dist_prev_node = rhs.dist_prev_node;
            lhs.anomalies = rhs.anomalies;
            lhs.color = rhs.color;
            lhs.number_of_edges = rhs.number_of_edges;
            lhs.angles_of_edges = rhs.angles_of_edges;
            lhs.position = rhs.position;
            
            lhs = super_node(lhs);            
        end
        
 
       
        function [t] = setType(obj, nodetype)
            obj.type = nodetype;
            t = obj;
        end
    end
end

\end{lstlisting}

\subsection{\emph{right\_bend.m}}

\begin{lstlisting}
%% Class defs of right-bend pipe
 
classdef right_bend < super_node
    properties
    end
    methods
        function [t] = right_bend(number, orientation, prev_node, diameter, dist_prev_node)
            
            %% Default constuctor
            if nargin == 0
                args = {};
            else
                if ~isempty(number)
                    args{1} = number;
                end
                if ~isempty(orientation)
                    args{2} = orientation;
                end
                if ~isempty(prev_node)
                    args{3} = prev_node;
                end
                if ~isempty(diameter)
                    args{4} = diameter;
                end
                if ~isempty(dist_prev_node)
                    args{5} = dist_prev_node;
                end
            end    
            %% calling super constructor.
            t = t@super_node(args{:});
            
            %% class specific
            
            t.color = 'red';
            t.type = 'Right Bend';
        end
        function [lhs] = super_node(rhs)
            %% converting function
            lhs.number = rhs.number;
            lhs.type = rhs.type;
            lhs.orientation = rhs.orientation;
            lhs.prev_node = rhs.prev_node;
            lhs.discovered = rhs.discovered;
            lhs.diameter = rhs.diameter;
            lhs.dist_prev_node = rhs.dist_prev_node;
            lhs.anomalies = rhs.anomalies;
            lhs.color = rhs.color;
            lhs.number_of_edges = rhs.number_of_edges;
            lhs.angles_of_edges = rhs.angles_of_edges;
            lhs.position = rhs.position;
            
            lhs = super_node(lhs);            
        end
        
    end
end
 

\end{lstlisting}


\section{\emph{sensorInterpreter.m}}

\begin{lstlisting}
%% Sensor Interpreter
%
% This is the class object which is resposible for recognizing the sensor
% output. 
% Output form this class is:    - Node Type
%                               - Position and Orientation
%                               - Distance traveled
%                               - Velocity
%                               - Distance to walls (?)
%                               - Register anomalies and irregualrities
 
 
classdef sensorinterpreter 
   
    properties(GetAccess = public, SetAccess = protected)
        
        %% Output data Whenever recognized
        recognized_node_struct; % A struct containing the data needed for the world object to draw the node.
        registered_anomalies_struct; % A struct containing details about the detected anomaly.
        
        %% Continous output data
        velocities; % size 3, 1 
        dist_to_walls; % size 2, 1 
        
        verden; % binding til verdenen over.
%     end
    
    %properties(GetAccess = private, SetAccess = private)
        %% Internal properties
        pipeline_profiles; % The profiles to match the sensor data to
    
        %% The sensor readings.
        LRF_data; % input: angles and ranges 2xN
        LRF_paramsy; % structs for holding data asociated with line fit
        LRF_paramsx; 
        LRF_height; % The height of the measurement plane.
        
%         FittedStructures_2D; % Structures for gathering 2d structures
%         FittedStructures_3D; % structures for gathering 3d Structures
        
        ToF_data; % input: point cloud, x,y,z Mx3
        ToF_params; % struct for the cylinder fits
        a0; % assumed direction of the pipe. 
        interval; %tof cylinder fit interval
        
        Stereo_data; % input: x,y,z of features
    
    end
    
    methods
        %% Constructor
        
        function object = sensorinterpreter(args)
           
            if nargin < 1 % default constuctor
                object.velocities = zeros(3,1);
                object.dist_to_walls = zeros(2,1);
                
                
                object.LRF_paramsy = struct('x0_urg', [], 'a_urg' ,[],...
                    'd_urg' , [], 'normd_urg', [],'x_urg' , [],'y_urg' , [],...
                    'histrange', []);
                
                object.LRF_paramsx = struct('x0_urg', [], 'a_urg' ,[],...
                    'd_urg' , [], 'normd_urg', [],'x_urg' , [],'y_urg' , [],...
                    'histrange', []);
            
                object.ToF_params = struct('x0k', [], 'ank',  [], 'rk', [],...
                    'dk', [], 'length_d', [], 'a_parmk', [], 'x0', []);
                            
                object.LRF_height = 0.07; % 7 cm above ground.
                
            elseif nargin == 2
                
                
            end
        end
        
        %% Asociating sensors to the object.
        
        function this = setLRFData(this, LRF_data)
            if (size(LRF_data, 1) ~= 2) || isempty(LRF_data)
                error('The LRF-data should be on the form 2xM')
            else
                this.LRF_data = LRF_data;
            end
        end
        
        function this = setToFData(this, ToF_data)
            if (size(ToF_data, 2) ~= 3) || isempty(ToF_data) 
                error('The LRF-data should be on the form Mx3')
            else
                this.ToF_data = ToF_data; 
            end
        end
        
        
        %% Internal Calculation Function
function [this, radius, closestRadius, indexMin] = fuseSensors(this, threshold)
            % find pralell lines, start with lines assumed along the pipe.
            if ~isempty(this.LRF_paramsy.a_urg)
                directions = this.LRF_paramsy.a_urg;
                
                equalToLine1 = zeros(size(directions,1),1);
                
                %should start with line nearest to origin.
                
                %compare the first with the others
                for i = 2:size(directions, 1)
                    if compareValue(directions(i,1), directions(1,1), threshold) == 0
                        if compareValue(directions(i,2), directions(1,2), threshold) == 0
                            disp('line 1 and line %i are within the threshold')
                            equalToLine1(i) = 1;
                        else
                            disp('Not equal');
                        end
                    end
                end
                
                %then find the ones that are closes togheter
                
                distanceToLine1 = zeros(size(directions,1),1);
                distanceToLine1(1) = Inf;
                for i = 2:size(directions, 1)
                    if equalToLine1(i) == 1
                        %calculat the distance from centeroids at x=0
                        distanceToLine1(i) = norm([0, this.LRF_paramsy.x0_urg(i,2)]-...
                            [0, this.LRF_paramsy.x0_urg(1,2)]);
                    end
                end
                
                %find the smalles distance assume this is at the center of the
                %pipe.
                [minDist, indexMin] = min(distanceToLine1);
                if minDist == 0
                    disp('No paralell lines found');
                    radius = 0;
                    closestRadius = 0;
                    indexMin = 0;
                else
                    disp('Found paralell lines');
                    radius = this.LRF_height/2 + (minDist^2)/(8*this.LRF_height);
                    indexMin
                    distanceToLine1
                    
                    closestRadius = inf;
                    
                    %compare the distance to the calculated radiuses.
                    for i = 1:size(this.ToF_params.rk,1)
                        switch compareValue(this.ToF_params.rk(i), radius, threshold/10)
                            case 0
                                disp('Found a radius whitihn 10 % of threshold of radius');
                                
                                closestRadius = this.ToF_params.rk(i);
                            case -1
                                disp('Found a radius smaller than radius');
                                if abs(radius - this.ToF_params.rk(i)) < 0.03
                                    closestRadius = this.ToF_params.rk(i);
                                end
                            case 1
                                disp('Found a radius larger than radius')
                                if abs(this.ToF_params.rk(i) - radius) < 0.03
                                    closestRadius = this.ToF_params.rk(i);
                                end
                            otherwise
                                disp('Unkonwn error')
                        end
                    end
                    
                    % Add the cylinder to the back of the ToF_params.
                    
                    % find direction
                    
                    
                    
                    direction = [ (this.LRF_paramsy.a_urg(1,2)+this.LRF_paramsy.a_urg(indexMin,2)/2), 0,...
                        -(this.LRF_paramsy.a_urg(1,1)+this.LRF_paramsy.a_urg(indexMin,1)/2)];
                    x0 = [0, this.LRF_height/2,0.2];
                 
                    [X, Y, Z] = cylinder(radius*ones(2,1), 125); % plot the cone.
                    
                    
                    % find rotation axis and transformation matrix
                    rot_axis = cross(this.a0, direction')
                    if norm(rot_axis) > eps
                        rot_angle = asin(norm(rot_axis))
                        if(dot(this.a0, direction')< 0)
                            rot_angle = pi-rot_angle;
                        end
                    else
                        rot_axis = this.a0
                        rot_angle = 0
                    end
                    
                    Rxyz = makehgtform('axisrotate', rot_axis, rot_angle);
                    Txyz = makehgtform('translate', x0');
                    
                    tf = Txyz*Rxyz;
                    
                    Xny = zeros(2, size(X,2));
                    Yny = zeros(2, size(Y,2));
                    Zny = zeros(2, size(Z,2));
                    % Transform cylinder to right scale and position
                    for i = 1:126
                        for j = 1:2
                            Xny(j, i) = (tf(1, 1:4)*[X(j,i); Y(j,i); Z(j,i); 1]);
                            Yny(j, i) = (tf(2, 1:4)*[X(j,i); Y(j,i); Z(j,i); 1]);
                            Zny(j, i) = (tf(3, 1:4)*[X(j,i); Y(j,i); Z(j,i); 1]);
                        end
                    end
                end
                figure;
                set(gcf, 'Renderer', 'opengl');
                plot3(this.ToF_data(:,3), this.ToF_data(:,1), this.ToF_data(:,2), '.');
                hold on;
                if minDist ~= 0
                    surface(Zny./tf(4,4), Xny./tf(4,4), Yny./tf(4,4));
                end
                hold off;
                axis equal;
                xlabel('Depth');
                ylabel('Camera X-direction');
                zlabel('Camera Y-direction');
                grid on
            end
        
            
end
 
  
  function type = matchPipeProfile(this)
            
  end
        
        
        %% find lines in 2d data
        function this = find2Dlines(this, number_points_x, number_points_y,...
                 histogram_interval_x, histogram_interval_y)
            
            if nargin ~= 5
                error('Too few arguments. Usage: find2DLines(num_point_x, num_points_x, histx_int, hitsy_int')
            else
                % start interpreting the data.
                %transform to cartesian coordinates
                [urgx, urgy] = pol2cart(this.LRF_data(1,:), this.LRF_data(2, :));
                
                % sort on x descending order
                sorted = sortrows([urgx', urgy'], -1);
                
                temp = sorted;
                temp(~any(sorted,2),:)=[]; %% remove trivial points (0,0)
                
                %form histogram ranges and calculate histograms
                this.LRF_paramsy.histrange = (-4.6:histogram_interval_y:4.7)';
                this.LRF_paramsx.histrange = (-4.6:histogram_interval_x:4.7)';
                
                [nx] = hist(temp(:,1), this.LRF_paramsx.histrange);
                [ny] = hist(temp(:,2), this.LRF_paramsy.histrange);
                
                for k = 1:size(this.LRF_paramsy.histrange)
                    
                    threshold = this.LRF_paramsy.histrange(k);
                    % horizontal lines. from the parallell to the y-axis
                    
                    data = [];
                    if ny(k) > number_points_y
                        for i = 1:size(temp(:,2))
                            if (temp(i,2) > threshold-histogram_interval_y) && ...
                                    (temp(i,2) < threshold+histogram_interval_y)
                                data = [data; temp(i,:)];
                            end
                        end
                    end
                    
                    %look for different shapes, arcs, lines, parallell to the y-axis
                    if (~isempty(data)) && (size(data, 1) > 2)
                        [x0_urgt, a_urgt, d_urgt, normd_urgt] = ls2dline(data);
                        
                        this.LRF_paramsy.x0_urg = [this.LRF_paramsy.x0_urg; x0_urgt'];
                        this.LRF_paramsy.a_urg = [this.LRF_paramsy.a_urg; a_urgt'];
                        this.LRF_paramsy.d_urg = [this.LRF_paramsy.d_urg; zeros(50,1); d_urgt]; %%adding zeros to see where the new line starts
                        this.LRF_paramsy.normd_urg =[this.LRF_paramsy.normd_urg; normd_urgt];
                        
                        %calculate the line
                        
                        %scaling
                        startline = -x0_urgt(1)+ min(data(:,1));
                        t = sqrt((max(data(:,1))-min(data(:,1)))^2 + (max(data(:,2))-min(data(:,2)))^2) ;
                        
                        x_urgt_s = (x0_urgt(1) + ((a_urgt(1)*startline)));
                        y_urgt_s = (x0_urgt(2) + ((a_urgt(2)*startline)));
                        
                        x_urgt_l = (x0_urgt(1) + (a_urgt(1)*t));
                        y_urgt_l = (x0_urgt(2) + (a_urgt(2)*t));
                        
                        
                        % assign start and stop points to the y-struct
                        this.LRF_paramsy.x_urg = [this.LRF_paramsy.x_urg; [x_urgt_s, x_urgt_l]];
                        this.LRF_paramsy.y_urg = [this.LRF_paramsy.y_urg; [y_urgt_s, y_urgt_l]];
                    end
                end
                
                % look along the x-axis
                for k = 1:size(this.LRF_paramsx.histrange)
                    threshold = this.LRF_paramsx.histrange(k);
                    datax = [];
                    if nx(k) > number_points_x
                        for i = 1:size(temp(:,2))
                            if (temp(i,1) > threshold-histogram_interval_x) && ...
                                    (temp(i,1) < threshold+histogram_interval_x)
                                
                                datax = [datax; temp(i,:)];
                            end
                        end
                    end
                    if (~isempty(datax)) && (size(datax, 1) > 2)
                        [x0_urgtx, a_urgtx, d_urgtx, normd_urgtx] = ls2dline(datax);
                        this.LRF_paramsx.x0_urg = [this.LRF_paramsx.x0_urg; x0_urgtx'];
                        this.LRF_paramsx.a_urg = [this.LRF_paramsx.a_urg; a_urgtx'];
                        this.LRF_paramsx.d_urg = [this.LRF_paramsx.d_urg; zeros(50,1); d_urgtx];
                        this.LRF_paramsx.normd_urg =[this.LRF_paramsx.normd_urg; normd_urgtx];
                        
                        %calculate the line
                        
                        %scaling
                        startline = -x0_urgtx(2)+ min(datax(:,2));
                        t = sqrt((max(datax(:,1))-min(datax(:,1)))^2 + (max(datax(:,2))-min(datax(:,2)))^2) ;
                        
                        x_urgt_s = (x0_urgtx(1) + ((a_urgtx(1)*startline)));
                        y_urgt_s = (x0_urgtx(2) + ((a_urgtx(2)*startline)));
                        
                        x_urgt_l = (x0_urgtx(1) + (a_urgtx(1)*t));
                        y_urgt_l = (x0_urgtx(2) + (a_urgtx(2)*t));
                        
                        this.LRF_paramsx.x_urg = [this.LRF_paramsx.x_urg; [x_urgt_s, x_urgt_l]];
                        this.LRF_paramsx.y_urg = [this.LRF_paramsx.y_urg; [y_urgt_s, y_urgt_l]];
                    end
                end
            end
        end
        
        
        function this = find3Dcylinders(this, interval, a0)
 
            this.a0 = a0;
            this.interval = interval;
            
            % sort rows and remove trivial points
            this.ToF_data= sortrows(this.ToF_data, 3); % sort the vector on z value.
            temp = this.ToF_data;
            temp(~any(this.ToF_data,2),:)=[]; %% remove trivial points
            this.ToF_data = temp;
            
            % Calculate the interval
            startz = this.ToF_data(1,3);
            stopz = this.ToF_data(size(this.ToF_data, 1), 3);
            
            pieces = ceil((stopz-startz)/interval); % round upwards toward nearest integer to ensure all values
            
            
            for k = 1:pieces
                %% Select data from total selection
                
                temp = [];
                for i = 1:size(this.ToF_data,1) % might be optimized because of the sorted array
                    if (this.ToF_data(i, 3) >= interval*(k-1)) && (this.ToF_data(i, 3) <= interval*k)
                        temp = [temp; this.ToF_data(i, :)];
                    end
                end
                
                
                
                %% Start the surface fit
                
                this.ToF_params.x0 = [this.ToF_params.x0; 0,0,interval*(k-1)];
                
                % [x0n, an, phin, rn, d, sigmah, conv, Vx0n, Van, uphin, urn, ...
                % GNlog, a, R0, R] = lscone(temp, x0, a0, angle, radius, 0.1, 0.1);
                
                if (isempty(temp) ) || (size(temp,1) < 5)
                    warning('Too few points')
                else
                    % Start cylinder fit using gauss-newton
                    [x0n, an, rn, d, sigmah, conv, Vx0n, Van, urn, GNlog, a] = ...
                        lscylinder(temp, (this.ToF_params.x0(k, :))', a0,...
                        this.verden.pipe_diameter/2, .001, .001);
                    
                    this.ToF_params.x0k = [this.ToF_params.x0k; x0n'];
                    this.ToF_params.ank = [this.ToF_params.ank; an'];
                    this.ToF_params.rk = [this.ToF_params.rk; rn];
                    this.ToF_params.dk = [this.ToF_params.dk; d];
                    this.ToF_params.length_d = [this.ToF_params.length_d; size(d,1)];
                    this.ToF_params.a_parmk = [this.ToF_params.a_parmk; a'];
                end
                
            end
        end
        
        
        
        %% Output function
        
        % This shows what the view should be like with the previous sensor
        % readings. This should open a plot with the the given view.
        function [rot_axis, rot_angle] = showSynthesizedView(this)
            
            rot_axis = zeros(3, size(this.ToF_params.x0k, 1));
            rot_angle = zeros(size(this.ToF_params.x0, 1), 1);
            
            figure;
            set(gcf, 'Renderer', 'opengl');
            test = plot3(this.ToF_data(:,3), this.ToF_data(:,1), this.ToF_data(:,2), '.');
            hold on;
            for k = 1:size(this.ToF_params.x0k, 1)
                
                if this.ToF_params.rk(k) > 2*(this.verden.pipe_diameter)/2 || this.ToF_params.rk(k) < 0
                    warning('Radius errenous. Skipping...');
                else
                    %% start constructing the cylinder from the given parameters.
                    [X, Y, Z] = cylinder(this.ToF_params.rk(k)*ones(2,1), 125); % plot the cone.
                    
                    
                    % find rotation axis and transformation matrix
                    rot_axis(:, k) = cross(this.a0,this.ToF_params.ank(k,:)');
                    if norm(rot_axis(:,k)) > eps
                        rot_angle(k) = asin(norm(rot_axis(:,k)));
                        if(dot(this.a0, this.ToF_params.ank(k,:)')< 0)
                            rot_angle(k) = pi-rot_angle(k);
                        end
                    else
                        rot_axis(:,k) = this.a0;
                        rot_angle(k) = 0;
                    end
                    
                    Sz = makehgtform('scale',[1;1;this.interval] );
                    Txyz = makehgtform('translate', this.ToF_params.x0k(k,:));
                    Rxyz = makehgtform('axisrotate', rot_axis(:,k), rot_angle(k));
                    Tz = makehgtform('translate', [0;0;-(this.interval/2)]);
                    
                    tf = Txyz*Rxyz*Sz;
                    Xny = zeros(2, size(X,2));
                    Yny = zeros(2, size(Y,2));
                    Zny = zeros(2, size(Z,2));
                    
                    % Transform cylinder to right scale and position
                    for i = 1:size(X,2)
                        for j = 1:2
                            Xny(j, i) = (tf(1, 1:4)*[X(j,i); Y(j,i); Z(j,i); 1]);
                            Yny(j, i) = (tf(2, 1:4)*[X(j,i); Y(j,i); Z(j,i); 1]);
                            Zny(j, i) = (tf(3, 1:4)*[X(j,i); Y(j,i); Z(j,i); 1]);
                        end
                        
                    end
                    h(k) = surface(Zny./tf(4,4), Xny./tf(4,4), Yny./tf(4,4));
                    
                    
                end
                
            end
            
            hold off;
            axis equal;
            xlabel('Depth');
            ylabel('Camera X-direction');
            zlabel('Camera Y-direction');
            grid on
            
            figure;
            plot(this.ToF_params.dk);
            title('Distance of points from the fitted cylinder');
            hold on
            plot([this.ToF_params.length_d; size(this.ToF_params.dk,1)],...
                zeros(1, size(this.ToF_params.length_d,1)+1), '+r', 'MarkerSize', 10);
            hold off;
            
%             figure;
%             plot(this.ToF_params.dk, this.ToF_data(:,3)); % distribution of the points along the z-axis
%             xlabel('Length from point on cylinder [m]');
%             ylabel('Detpth into the pipe, Z-axis [m]');
%             grid on
%             
        end
        
        % this creates the real view from the sensors at the given moment.
        function [] = plot2Dlines(this)
            
            [x, y ]= pol2cart(this.LRF_data(1,:), this.LRF_data(2,:));
            figure;
            subplot(2, 2, 1:2);
            plot(x, y, 'b.');
            hold on;
            grid on;
            xlabel('Depth into the pipe Z-axis [m]');
            ylabel('X-axis relative to ToF-camera [m]');
            title('URG Laser Range Finder');
            
            for i = 1:size(this.LRF_paramsy.x_urg,1)
                line([this.LRF_paramsy.x_urg(i,1);this.LRF_paramsy.x_urg(i,2)],...
                    [this.LRF_paramsy.y_urg(i,1);this.LRF_paramsy.y_urg(i,2)], 'Color', 'magenta', ...
                    'LineWidth', 1.5);
                %plot centroid
                plot(this.LRF_paramsy.x0_urg(i,1), this.LRF_paramsy.x0_urg(i,2), 'k*');
            end
            
            for i = 1:size(this.LRF_paramsx.x_urg,1)
                line([this.LRF_paramsx.x_urg(i,1);this.LRF_paramsx.x_urg(i,2)],...
                    [this.LRF_paramsx.y_urg(i,1);this.LRF_paramsx.y_urg(i,2)], 'Color', 'red', ...
                    'LineWidth', 1.5);
                %plot centroid
                plot(this.LRF_paramsx.x0_urg(i,1), this.LRF_paramsx.x0_urg(i,2), 'g*');
            end
            hold off;
            
            
            subplot(2, 2, 3);
            plot(this.LRF_paramsy.d_urg);
            title('Error from the fitted lines Horizontal');
            grid on;
            
            subplot(2, 2, 4);
            plot(this.LRF_paramsx.d_urg);
            title('Error from the fitted lines Vertical');
            grid on;
            
            
            figure;
            subplot(1, 2, 1)
            hist(x, this.LRF_paramsx.histrange);
            title('Histogram of Vertical pixels');
            xlabel('Position depth into the pipe (camera z-axis) [m]');
            
            subplot(1, 2, 2);
            hist(y, this.LRF_paramsy.histrange);
            title('Histogram of Horizontal pixels');
            xlabel('Position lateral position into the pipe(camera x-axis) [m]');
            
            
        end
        
        
        %% Other functions
        
        function this = setWorld(this, world)
           this.verden = world; 
        end
        
        
    end
    
end
\end{lstlisting}


\chapter{\emph{C/C++ implementations}}

\section{Stereo Camera Interface}

\lstloadlanguages{[ISO]C++}

\lstset{% general command to set parameter(s)
    language=C++,
    basicstyle=\scriptsize, % print whole listing small
    stringstyle=\ttfamily, % typewriter type for strings
    showstringspaces=false} % no special string spaces


\subsection{denseStereo.cpp}

\begin{lstlisting}
/*
 * DenseStereo.cpp
 *
 *  Created on: 17. apr. 2010
 *      Author: anderga
 */


#include <iostream>
#include <string>
#include <cstdio>
#include <cstdlib>

#include <fstream>
#include <iomanip> // for manipulating streams



#include "StereoVision.h"
#include "StereoCamera.h"

#include <opencv/highgui.h>

int sampleTimeout;
CvMat* imageRectifiedPair;

std::string filename = "c:\\test2.txt";

void displayOutput(StereoCamera * camera, StereoVision * vision){

    CvSize imageSize = vision->getImageSize();
    if(!imageRectifiedPair)
    	imageRectifiedPair = cvCreateMat( imageSize.height, imageSize.width*2,CV_8UC3 );

    vision->stereoProcess(camera->getFramesGray(0), camera->getFramesGray(1), STEREO_MATCH_BY_BM);


    CvMat part;
    cvGetCols( imageRectifiedPair, &part, 0, imageSize.width );
    cvCvtColor( vision->imagesRectified[0], &part, CV_GRAY2BGR );
    cvGetCols( imageRectifiedPair, &part, imageSize.width,imageSize.width*2 );
    cvCvtColor( vision->imagesRectified[1], &part, CV_GRAY2BGR );

    for(int j = 0; j < imageSize.height; j += 16 )
    	cvLine( imageRectifiedPair, cvPoint(0,j),
    			cvPoint(imageSize.width*2,j),
    			CV_RGB((j%3)?0:255,((j+1)%3)?0:255,
    					((j+2)%3)?0:255));

    cvShowImage( "rectified", imageRectifiedPair );

    cvShowImage( "depth", vision->imageDepthNormalized );
}


int calibrationLoad(StereoVision* vision, std::string filename)
{
    std::cout << "Loading calibration file..." << filename;
    if(0 == vision->calibrationLoad(filename.c_str())){
        std::cout << " OK" << std::endl;
        return 0;
    }else{
        std::cout << " -FAIL" << std::endl;
        return 1;
    }
}

void calibrationSave(StereoVision* vision, std::string filename)
{
    std::cout << "Saving calibration file..." << filename;
    if(0 == vision->calibrationSave(filename.c_str())){
        std::cout << " OK" << std::endl;
    }else{
        std::cout <<" -FAIL"<< std::endl;
    }
}

void writeToFile(IplImage* _3dimage){

	//open the file

	FILE *File;

	File = fopen(filename.c_str(), "a");

	std::cout << "Check image atributes" << std::endl;

	if (_3dimage->depth == IPL_DEPTH_16S)
		std::cout << "Depth of image is signed 16 bit integer" << std::endl;
	else if (_3dimage->depth == IPL_DEPTH_32F)
		std::cout << "Depth of image is 32 bit float" << std::endl;
	else if (_3dimage->depth == IPL_DEPTH_8S)
		std::cout << "Depth of image is signed 8 bit integer" << std::endl;

	std::cout << "Number of channels in the image is " << _3dimage->nChannels << std::endl;

	if( _3dimage->dataOrder == IPL_DATA_ORDER_PIXEL)
		std::cout << "The dataorder of the image is Ordered Pixel" << std::endl;
	else if (_3dimage->dataOrder == IPL_DATA_ORDER_PLANE)
		std::cout << "The dataorder of the image is Ordered Plane" << std::endl;




	std::cout << "Starting to write file...";


	if (File != NULL){
		//access the image data.
		for (int y = 0; y < _3dimage->height; y++){
			float * ptr = (float*)(_3dimage->imageData + y*_3dimage->widthStep);
			for (int x = 0; x < _3dimage->width; x++){
/*
				File << fixed << std::setprecision(1) << (int)ptr[3*x] << "," <<
						fixed << std::setprecision(1) << (int)ptr[3*x+1] << "," <<
						fixed << std::setprecision(1) << (int)ptr[3*x+2];
				if ( x == _3dimage->width-1)
					File << " ";
				else
					File << ",";

					*/
				fprintf(File, "%f,%f,%f,", (float)ptr[3*x], (float)ptr[3*x+1], (float)ptr[3*x+2]);

			}
			fprintf(File, "\n");
		}
	}
	fclose(File);
	std::cout << " Done!" << std::endl;


}


void create3dOutput(StereoCamera *camera, StereoVision * vision, IplImage * dest){
	CvSize imageSize = vision->getImageSize();
	//if(!imageRectifiedPair)
		//imageRectifiedPair = cvCreateMat( imageSize.height, imageSize.width*2,CV_8UC3 );

	//process the stereo
	//vision->stereoProcess(camera->getFramesGray(0), camera->getFramesGray(1), STEREO_MATCH_BY_BM);

	//cvShowImage("left", camera->frames[0]);

	if (vision->Q == 0){
		std::cout << "Q matrix not set."<< std::endl;
		cvSetIdentity(vision->Q);
	}else if( vision->imageDepth == 0){
		std::cout << "Depth Image not set" << std::endl;
	}else{

		CvMat *realDisp = cvCreateMat(vision->imageDepth->height,
				vision->imageDepth->width,
				CV_16SC1);
		cvConvertScale( vision->imageDepth, realDisp, 1.0/16, 0 );

		cvReprojectImageTo3D(realDisp, dest, vision->Q, true);
//		cvSave("c:\\reprojection_matrix_Q.xml", vision->Q);
//		cvSave("c:\\resultImage3d.xml", dest);

		writeToFile(dest);
	}

	//cvShowImage( "depth", dest );
}


int main(int argc, char* argv[]){

	bool usefile = false;
	bool saveCalibration = false;

	bool writefile = false;
	int iterator = 0;

	CvMat C1, C2, D1, D2, F;

	IplImage *img3d = 0;

	CvSize resolution = cvSize(640,480);

	StereoCamera *camera = new StereoCamera(resolution);

    StereoVision* vision = new StereoVision(resolution);

	int cornersX, cornersY;

	cornersX = 8;
	cornersY = 5;

	int result;

	while(1){

		if(!camera->ready){
			std::cout <<"Connecting to cameras..." << std::endl;
			if(0 != camera->setup(resolution)){
				std::cout << "-FAILED" << std::endl;
			}else{
				std::cout <<"...OK" << std::endl;

				cvNamedWindow( "left");
				cvNamedWindow( "right");
				cvNamedWindow( "rectified", 1 );
				cvNamedWindow( "depth", 1 );
				vision->calibrationStart(cornersX, cornersY);
			}
		}else{
			if(0 == camera->capture()){
				if(!vision->getCalibrationDone()){
					cvShowImage("left",camera->frames[0]);
					cvShowImage("right",camera->frames[1]);
				}
				int c = cvWaitKey( 1 );
					if( c == 27 ) //if esc key break.
						break;
					else if (c == 'c'){
						vision->reCalibrate(cornersX, cornersY);
						std::cout << "Recalibrate" << std::endl;
					}else if (c == 's'){
						writefile = true;
						std::cout << "User chose to write file"<< std::endl;
					}else if (c == 'l'){
						std::cout << "Loading calibration... " << std::endl;
						calibrationLoad(vision, "calibration1.xml");
						std::cout << "Loading calibration... Done!" << std::endl;
					}
			}


			if(vision->getCalibrationStarted()){//start the calibration process to rectify the images

				if (usefile ){
					if(calibrationLoad(vision, "calibration.xml") == 0){
						std::cout << "Calibration Done !" << std::endl;
						usefile = false;
					}
				}else{

					IplImage * left_gray = camera->getFramesGray(0);
					IplImage * right_gray = camera->getFramesGray(1);

					result = vision->calibrationAddSample(left_gray, right_gray);

					if(0 == result){
						std::cout << "+OK" << std::endl;
						if(vision->getSampleCount() >= 30){
							vision->calibrationEnd(
									STEREO_CALIBRATE_BOTH_CAMERAS,
									&D1, &C1, &D2, &C2, &F);
							std::cout << "Calibration Done !" << std::endl;
							if(saveCalibration)
								calibrationSave(vision, "calibration");
						}
					}else{
						std::cout <<"-FAIL Try a different position. Chessboard should be visible on both cameras." << std::endl;
					}
				}

			}else{ // Display the depth output.
				if(vision->getCalibrationDone()){
					displayOutput(camera, vision);

					if (writefile && iterator <= 4){
						if (img3d == 0)
							img3d = cvCreateImage( resolution, IPL_DEPTH_32F, 3 );

						create3dOutput(camera, vision, img3d);
						iterator++;
					}
					if (iterator > 4){
						writefile = false;
						iterator = 0;
					}


				}
			}


		}
	}



	cvDestroyAllWindows();
	cvReleaseMat(&imageRectifiedPair);
	cvReleaseImage(&img3d);



    delete vision;
    delete camera;

    return 0;
}
\end{lstlisting}

\subsection{StereoCamera.h}

\begin{lstlisting}
/*
 * StereoCamera.h
 *
 *  Created on: 17. apr. 2010
 *      Author: anderga
 */

#ifndef STEREOCAMERA_H_
#define STEREOCAMERA_H_

#include "opencv/cv.h"
#include "opencv/cxmisc.h"
#include "opencv/cvaux.h"
#include "opencv/highgui.h"



class StereoCamera
{
    CvCapture* captures[2];

    CvSize imageSize;

public:

    IplImage* frames[2];
    IplImage* framesGray[2];

    //IplImage* render[2];

    CvMat D1, D2, R1, R2; //used for parameter output and saving of estimated paramters.


    StereoCamera();
    StereoCamera(CvSize size);
    ~StereoCamera();

    int setup(CvSize imageSize);
    bool ready;
    int capture();
    IplImage* getFramesGray(int lr);

};


#endif /* STEREOCAMERA_H_ */

\end{lstlisting}


\subsection{StereoCamera.cpp}

\begin{lstlisting}
/*
 * StereoCamera.cpp
 *
 *  Created on: 17. apr. 2010
 *      Author: anderga
 */

#include "StereoCamera.h"


StereoCamera::StereoCamera()
{
    for(int lr=0;lr<2;lr++){
        captures[lr] = 0;
        frames[lr] = 0;
        framesGray[lr] =0;

    }
    ready = false;
}

StereoCamera::StereoCamera(CvSize size)
{
    for(int lr=0;lr<2;lr++){
        captures[lr] = 0;
        frames[lr] = 0;
        framesGray[lr] =0;

    }
    imageSize = size;

    ready = false;
}



StereoCamera::~StereoCamera()
{
    for(int lr=0;lr<2;lr++){
        cvReleaseImage(&frames[lr]);
        cvReleaseImage(&framesGray[lr]);
        cvReleaseCapture(&captures[lr]);
        //cvReleaseImage(&render[lr]);
    }

}

int StereoCamera::setup(CvSize imageSize){
    this->imageSize = imageSize;

    captures[0] = cvCaptureFromCAM(0);
    captures[1] = cvCaptureFromCAM(1);

/*    frames[0] = cvCreateImage(imageSize,IPL_DEPTH_8U,3);
    framesGray[0] = cvCreateImage(imageSize,IPL_DEPTH_8U,1);
    frames[1] = cvCreateImage(imageSize,IPL_DEPTH_8U,3);
    framesGray[1] = cvCreateImage(imageSize,IPL_DEPTH_8U,1);
*/
    if( (captures[0]) && (captures[1])){

        for(int i=0;i<2;i++){
                cvSetCaptureProperty(captures[i] ,CV_CAP_PROP_FRAME_WIDTH,imageSize.width);
                cvSetCaptureProperty(captures[i] ,CV_CAP_PROP_FRAME_HEIGHT,imageSize.height);
        }


        ready = true;
        return 0;
    }else{
        ready = false;
        return 1;
    }

}

int StereoCamera::capture(){


    frames[0] = cvQueryFrame(captures[0]);
    frames[1]= cvQueryFrame(captures[1]);

    //render[0] = cvCloneImage(frames[0]);
    //render[1] = cvCloneImage(frames[1]);


    return (captures[0] && captures[1]) ? 0 : 1;
}

IplImage*  StereoCamera::getFramesGray(int lr){
    if(!frames[lr])
    	return 0;
    if(frames[lr]->depth == 1){
        framesGray[lr] = frames[lr];
        return frames[lr];
    }else{
        if(framesGray[lr] == 0)
        	framesGray[lr] = cvCreateImage(cvGetSize(frames[lr]),IPL_DEPTH_8U,1);
        cvCvtColor(frames[lr],framesGray[lr],CV_BGR2GRAY);
        return framesGray[lr];
    }
}

\end{lstlisting}



\subsection{StereoVision.h}

\begin{lstlisting}
/*
 * StereoVision.h
 *
 *  Created on: 17. apr. 2010
 *      Author: anderga
 */

#ifndef STEREOVISION_H_
#define STEREOVISION_H_

#include <opencv/cv.h>
#include <opencv/cxmisc.h>
#include <opencv/cvaux.h>
#include <opencv/highgui.h>

using namespace std;
#include <vector>

enum{
	STEREO_CALIBRATE_BOTH_CAMERAS = 0,
	STEREO_CALIBRATE_INDIVIDUAL_CAMERAS = 1,
	STEREO_MATCH_BY_BM = 2,
	STEREO_MATCH_BY_GC = 3
};


class StereoVision
{
private:
    //chesboard board corners X,Y, N = X*Y ,  number of corners = (number of cells - 1)
    int cornersX,cornersY,cornersN;
    int sampleCount;
    bool calibrationStarted;
    bool calibrationDone;

    CvSize imageSize;


    vector<CvPoint2D32f> ponintsTemp[2];
    vector<CvPoint3D32f> objectPoints;
    vector<CvPoint2D32f> points[2];
    vector<int> npoints;

public:
    StereoVision(int imageWidth,int imageHeight);
    StereoVision(CvSize size);
    ~StereoVision();

    //matrices resulting from calibration (used for cvRemap to rectify images)
    CvMat *mx1,*my1,*mx2,*my2;

    CvMat * Q; //perspective transformation matrix created by stereoRectify

    CvMat* imagesRectified[2];
    CvMat  *imageDepth,*imageDepthNormalized;

    CvMat * disp_gc[2];


    void calibrationStart(int cornersX,int cornersY);
    int calibrationAddSample(IplImage* imageLeft,IplImage* imageRight);
    int calibrationEnd(int flag, CvMat* dist1, CvMat* cam1, CvMat* dist2, CvMat* cam2,CvMat* fundamentalMat );

    int calibrationSave(const char* filename);
    int calibrationLoad(const char* filename);

    int stereoProcess(CvArr* imageSrcLeft,CvArr* imageSrcRight, int match);

    CvSize getImageSize(){return imageSize;}
    bool getCalibrationStarted(){return calibrationStarted;}
    bool getCalibrationDone(){return calibrationDone;}
    int getSampleCount(){return sampleCount;}

    void reCalibrate(int cornersX, int cornersY);

};


#endif /* STEREOVISION_H_ */

\end{lstlisting}

\subsection{StereoVision.cpp}

\begin{lstlisting}

/*
 * StereoVision.cpp
 *
 *  Created on: 17. apr. 2010
 *      Author: anderga
 */

#include "StereoVision.h"

#include <iostream>
#include <string>

StereoVision::StereoVision(int imageWidth,int imageHeight)
{
    this->imageSize = cvSize(imageWidth,imageHeight);
    mx1 = my1 = mx2 = my2 = 0;
    Q = 0;
    calibrationStarted = false;
    calibrationDone = false;
    imagesRectified[0] = imagesRectified[1] = imageDepth = imageDepthNormalized = 0;
    sampleCount = 0;
    disp_gc[0] = disp_gc[1] = 0;
}

StereoVision::StereoVision(CvSize size){

	this->imageSize = size;
	mx1 = my1 = mx2 = my2 = 0;
	Q = 0;
	calibrationStarted = false;
	calibrationDone = false;
	imagesRectified[0] = imagesRectified[1] = imageDepth = imageDepthNormalized = 0;
	imageDepth = 0;
	sampleCount = 0;
	disp_gc[0] = disp_gc[1] = 0;
}

StereoVision::~StereoVision()
{
    cvReleaseMat(&imagesRectified[0]);
    cvReleaseMat(&imagesRectified[1]);
    cvReleaseMat(&imageDepth);
    cvReleaseMat(&imageDepthNormalized);
    cvReleaseMat(&disp_gc[0]);
    cvReleaseMat(&disp_gc[1]);
}

void StereoVision::calibrationStart(int cornersX,int cornersY){
    this->cornersX = cornersX;
    this->cornersY = cornersY;
    this->cornersN = cornersX*cornersY;
    ponintsTemp[0].resize(cornersN);
    ponintsTemp[1].resize(cornersN);
    sampleCount = 0;
    calibrationStarted = true;
}

void StereoVision::reCalibrate(int cornersX, int cornersY){
	calibrationStarted = false;
	calibrationDone = false;
	calibrationStart(cornersX, cornersY);

}


int StereoVision::calibrationAddSample(IplImage* imageLeft,IplImage* imageRight){

    if(!calibrationStarted) return 1;

    IplImage* image[2] = {imageLeft,imageRight};

    int succeses = 0;

    for(int lr=0;lr<2;lr++){
        CvSize imageSize =  cvGetSize(image[lr]);

         if(imageSize.width != this->imageSize.width || imageSize.height != this->imageSize.height)
            return 1;

        int cornersDetected = 0;

        printf("Processing sample %d of 2 Finding chessboard corners ...\n", lr+1) ;

		// Find chessboard corners:
		int result = cvFindChessboardCorners(
				image[lr],
				cvSize(cornersX, cornersY),
				&ponintsTemp[lr][0],
				&cornersDetected,
				CV_CALIB_CB_ADAPTIVE_THRESH | CV_CALIB_CB_FILTER_QUADS );

		// Get subpixel accuracy on those corners

		cvFindCornerSubPix(
				image[lr],
				&ponintsTemp[lr][0],
				cornersDetected,
				cvSize( 11, 11 ),
				cvSize( -1, -1 ),
				cvTermCriteria( CV_TERMCRIT_EPS+CV_TERMCRIT_ITER, 30, 0.1 ));


        if(result && (cornersDetected == cornersN)){
            succeses++;
        }

    }
    if(succeses == 2){
        for(int lr=0;lr<2;lr++){
            points[lr].resize((sampleCount+1)*cornersN);
            copy( ponintsTemp[lr].begin(), ponintsTemp[lr].end(),  points[lr].begin() + sampleCount*cornersN);
        }
        sampleCount++;
        return 0;
    }else{
        return 1;
    }
}

int StereoVision::calibrationEnd(int flag,
				CvMat* dist1, //output of distrotion paramters for cam1
				CvMat* cam1,  // output of camera matrix for cam1
				CvMat* dist2,
				CvMat* cam2,
				CvMat* fundamentalMat){

    calibrationStarted = false;

    int useUncalibrated = 0;

    // ARRAY AND VECTOR STORAGE:
    double M1[3][3], M2[3][3], D1[5], D2[5];
    double R[3][3], T[3], E[3][3], F[3][3];
    CvMat _M1,_M2,_D1,_D2,_R,_T,_E,_F;

    _M1 = cvMat(3, 3, CV_64F, M1 );
    _M2 = cvMat(3, 3, CV_64F, M2 );
    _D1 = cvMat(1, 5, CV_64F, D1 );
    _D2 = cvMat(1, 5, CV_64F, D2 );
    _R = cvMat(3, 3, CV_64F, R );
    _T = cvMat(3, 1, CV_64F, T );
    _E = cvMat(3, 3, CV_64F, E );
    _F = cvMat(3, 3, CV_64F, F );

    // HARVEST CHESSBOARD 3D OBJECT POINT LIST:
    objectPoints.resize(sampleCount*cornersN);

    for(int k=0;k<sampleCount;k++)
        for(int i = 0; i < cornersY; i++ )
            for(int j = 0; j < cornersX; j++ )
                objectPoints[k*cornersY*cornersX + i*cornersX + j] = cvPoint3D32f(i, j, 0);


    npoints.resize(sampleCount,cornersN);

    int N = sampleCount * cornersN;


    CvMat _objectPoints = cvMat(1, N, CV_32FC3, &objectPoints[0] );
    CvMat _imagePoints1 = cvMat(1, N, CV_32FC2, &points[0][0] );
    CvMat _imagePoints2 = cvMat(1, N, CV_32FC2, &points[1][0] );
    CvMat _npoints = cvMat(1, npoints.size(), CV_32S, &npoints[0] );
    cvSetIdentity(&_M1);
    cvSetIdentity(&_M2);
    cvZero(&_D1);
    cvZero(&_D2);


    switch(flag){
    case STEREO_CALIBRATE_BOTH_CAMERAS:
    	//CALIBRATE THE STEREO CAMERAS
    	cvStereoCalibrate( &_objectPoints, &_imagePoints1,
    			&_imagePoints2, &_npoints,
    			&_M1, &_D1, &_M2, &_D2,
    			imageSize, &_R, &_T, &_E, &_F,
    			cvTermCriteria(CV_TERMCRIT_ITER+CV_TERMCRIT_EPS, 100, 1e-5),
    			CV_CALIB_FIX_ASPECT_RATIO + CV_CALIB_ZERO_TANGENT_DIST + CV_CALIB_SAME_FOCAL_LENGTH
    	);
    	break;
    case STEREO_CALIBRATE_INDIVIDUAL_CAMERAS:
    	//calibrate left camera
    	cvCalibrateCamera2( &_objectPoints, &_imagePoints1,
    			&_npoints, imageSize,
    			&_M1, &_D1, NULL, NULL,
    			0
    	);
    	cvCalibrateCamera2( &_objectPoints, &_imagePoints2,
    			&_npoints, imageSize,
    			&_M2, &_D2, NULL, NULL,
    			0
    	);
    	cvStereoCalibrate( &_objectPoints, &_imagePoints1,
    			&_imagePoints2, &_npoints,
    			&_M1, &_D1, &_M2, &_D2,
    			imageSize, &_R, &_T, &_E, &_F,
    			cvTermCriteria(CV_TERMCRIT_ITER+CV_TERMCRIT_EPS, 100, 1e-5),
    			CV_CALIB_FIX_INTRINSIC
    	);
    	break;
    default:
    	break;
    }

    //Always work in undistorted space
    cvUndistortPoints( &_imagePoints1, &_imagePoints1,&_M1, &_D1, 0, &_M1 );
    cvUndistortPoints( &_imagePoints2, &_imagePoints2,&_M2, &_D2, 0, &_M2 );

    //COMPUTE AND DISPLAY RECTIFICATION

	cvReleaseMat(&mx1);
	cvReleaseMat(&my1);
	cvReleaseMat(&mx2);
	cvReleaseMat(&my2);
	mx1 = cvCreateMat( imageSize.height,imageSize.width, CV_32F );
	my1 = cvCreateMat( imageSize.height,imageSize.width, CV_32F );
	mx2 = cvCreateMat( imageSize.height,imageSize.width, CV_32F );
	my2 = cvCreateMat( imageSize.height,imageSize.width, CV_32F );

    double R1[3][3], R2[3][3], P1[3][4], P2[3][4];
    CvMat _R1 = cvMat(3, 3, CV_64F, R1);
    CvMat _R2 = cvMat(3, 3, CV_64F, R2);

    if(useUncalibrated == 0){
    	CvMat _P1 = cvMat(3, 4, CV_64F, P1);
    	CvMat _P2 = cvMat(3, 4, CV_64F, P2);

    	if(!Q) // create the perspective transformation
    		Q = cvCreateMat(4, 4, CV_32F);

    	cvStereoRectify( &_M1, &_M2, &_D1, &_D2, imageSize,
    			&_R, &_T,
    			&_R1, &_R2, &_P1, &_P2, Q,
    			0/*CV_CALIB_ZERO_DISPARITY*/ );
    	//Precompute maps for cvRemap()
    	cvInitUndistortRectifyMap(&_M1,&_D1,&_R1,&_P1,mx1,my1);
    	cvInitUndistortRectifyMap(&_M2,&_D2,&_R2,&_P2,mx2,my2);

    }else{

    	//HARTLEY'S RECTIFICATION METHOD
    	double H1[3][3], H2[3][3], iM[3][3];
    	CvMat _H1 = cvMat(3, 3, CV_64F, H1);
    	CvMat _H2 = cvMat(3, 3, CV_64F, H2);
    	CvMat _iM = cvMat(3, 3, CV_64F, iM);

    	cvStereoRectifyUncalibrated(
    			&_imagePoints1,&_imagePoints2, &_F,
    			imageSize,
    			&_H1, &_H2, 3
    	);
    	cvInvert(&_M1, &_iM);
    	cvMatMul(&_H1, &_M1, &_R1);
    	cvMatMul(&_iM, &_R1, &_R1);
    	cvInvert(&_M2, &_iM);
    	cvMatMul(&_H2, &_M2, &_R2);
    	cvMatMul(&_iM, &_R2, &_R2);


    	cvInitUndistortRectifyMap(&_M1,&_D1,&_R1,&_M1,mx1,my1);
    	cvInitUndistortRectifyMap(&_M2,&_D2,&_R2,&_M2,mx2,my2);
    }
    calibrationDone = true;

    // copy the paramtere matrices
    cvInitMatHeader(dist1, 5, 1, CV_64F, D1);
    cvInitMatHeader(cam1, 3, 3, CV_64F, M1);
    cvInitMatHeader(dist2, 5, 1, CV_64F, D2);
    cvInitMatHeader(cam2, 3, 3, CV_64F, M2);
    cvInitMatHeader(fundamentalMat, 3, 3, CV_64F, F);

    return 0;
}

int StereoVision::stereoProcess(CvArr* imageSrcLeft,CvArr* imageSrcRight, int match){
    if(!calibrationDone)
    	return 1;

    if(!imagesRectified[0])
    	imagesRectified[0] = cvCreateMat( imageSize.height,imageSize.width, CV_8U );
    if(!imagesRectified[1])
    	imagesRectified[1] = cvCreateMat( imageSize.height,imageSize.width, CV_8U );

    //rectify images
    cvRemap( imageSrcLeft, imagesRectified[0] , mx1, my1 );
    cvRemap( imageSrcRight, imagesRectified[1] , mx2, my2 );

    CvStereoBMState *BMState = cvCreateStereoBMState();
    CvStereoGCState * GCState = cvCreateStereoGCState(16, 2);

    //chose algorithm
    switch(match) {
    case STEREO_MATCH_BY_BM:
    	if(!imageDepth)
   	    	imageDepth = cvCreateMat( imageSize.height,imageSize.width, CV_16S );
  	    if(!imageDepthNormalized)
   	    	imageDepthNormalized = cvCreateMat( imageSize.height,imageSize.width, CV_8U );


    	BMState->preFilterSize=41;
    	BMState->preFilterCap=31;
    	BMState->SADWindowSize=41;
    	BMState->minDisparity=-64;
    	BMState->numberOfDisparities=128;
    	BMState->textureThreshold=10;
    	BMState->uniquenessRatio=15;

    	cvFindStereoCorrespondenceBM( imagesRectified[0],
    			imagesRectified[1],
    			imageDepth,
    			BMState);
    	cvNormalize( imageDepth, imageDepthNormalized, 0, 256, CV_MINMAX );


    	cvReleaseStereoBMState(&BMState);
    	break;

    case STEREO_MATCH_BY_GC:

    	if(!disp_gc[0])
   	    	disp_gc[0] = cvCreateMat( imageSize.height,imageSize.width, CV_16S );
  	    if(!disp_gc[1])
   	    	disp_gc[1] = cvCreateMat( imageSize.height,imageSize.width, CV_16S );



  	    cvFindStereoCorrespondenceGC(imagesRectified[0],
  	    		imagesRectified[1],
  	    		disp_gc[0],
  	    		disp_gc[1],
  	    		GCState,
  	    		0);

  	    if(!imageDepthNormalized)
  	       	imageDepthNormalized = cvCreateMat( imageSize.height,imageSize.width, CV_8U );


  	    cvConvertScale(disp_gc[0], imageDepthNormalized, -16);

  	    cvReleaseStereoGCState(&GCState);
  	    break;
    }



    return 0;
}

int StereoVision::calibrationSave(const char* filename){
    if(!calibrationDone)
    	return 1;


    FILE* f =  fopen(filename,"wb");
    if(!f)
    	return 1;
    if(!fwrite(mx1->data.fl,sizeof(float),mx1->rows*mx1->cols,f))
    	return 1;
    if(!fwrite(my1->data.fl,sizeof(float),my1->rows*my1->cols,f))
    	return 1;
    if(!fwrite(mx2->data.fl,sizeof(float),mx2->rows*mx2->cols,f))
    	return 1;
    if(!fwrite(my2->data.fl,sizeof(float),my2->rows*my2->cols,f))
    	return 1;
    fclose(f);


    return 0;
}


int StereoVision::calibrationLoad(const char* filename){
	calibrationStarted = false;
    cvReleaseMat(&mx1);
    cvReleaseMat(&my1);
    cvReleaseMat(&mx2);
    cvReleaseMat(&my2);
    mx1 = cvCreateMat( imageSize.height,imageSize.width, CV_32F );
    my1 = cvCreateMat( imageSize.height,imageSize.width, CV_32F );
    mx2 = cvCreateMat( imageSize.height,imageSize.width, CV_32F );
    my2 = cvCreateMat( imageSize.height,imageSize.width, CV_32F );
    FILE* f =  fopen(filename,"rb");
    if(!f) return 1;
    if(!fread(mx1->data.fl,sizeof(float),mx1->rows*mx1->cols,f)) return 1;
    if(!fread(my1->data.fl,sizeof(float),my1->rows*my1->cols,f)) return 1;
    if(!fread(mx2->data.fl,sizeof(float),mx2->rows*mx2->cols,f)) return 1;
    if(!fread(my2->data.fl,sizeof(float),my2->rows*my2->cols,f)) return 1;
    fclose(f);
    calibrationDone = true;
    return 0;
}

\end{lstlisting}

\section{Hokuyo URG Matlab Interface}

\begin{lstlisting}


#include <mex.h>
#include <stdio.h>
#include <stdlib.h>
#include <string.h>


#include "urg/urg_ctrl.h"


void urg_exit(urg_t *urg, const char *message)
{
	printf("%s: %s\n", message, urg_error(urg));
	urg_disconnect(urg);

	return;
}

void createEmptyOutputMatrices(int nlhs, mxArray * plhs[])
{
    int i;
    for (i=0; i<nlhs;i++)
    {
        plhs[i] = mxCreateDoubleMatrix(0, 0, mxREAL);
    }
}


int createOutput(urg_t * urg,
		unsigned long * data,
		unsigned long timestamp,
		unsigned int n_points,
		int nlhs,
		mxArray * plhs[],
		int giveIntensity)
{
	int numOutArgs;
	if(giveIntensity)
		numOutArgs =  2; //getNumOutputArgs(sensorType, scanType);
	else
		numOutArgs = 1;

    if (numOutArgs <= 0)
    {
        createEmptyOutputMatrices(nlhs,plhs);
    }

    int maxNumPoints = urg_dataMax(urg);


    if (n_points > maxNumPoints)
    {
        printf("createOutput: ERROR:too many points received = %d,\n", n_points);
        createEmptyOutputMatrices(nlhs,plhs);
        return -1;
    }

    if (n_points % numOutArgs != 0)
    {
        printf("createOutput: ERROR: (number of data points) mod (number of output vars) is not zero!! n_points=%d, n_vars=%d\n",
        		n_points, numOutArgs);
        createEmptyOutputMatrices(nlhs,plhs);
        return -1;
    }

    mxArray *out0, *out1, *out2;
    int i;
    switch (numOutArgs)
    {
        case 1:
            out0=mxCreateDoubleMatrix(n_points,1,mxREAL);
            for (i=0;i<n_points;i++)
            {
            	if (i <= 44){
            		mxGetPr(out0)[i] = 0.0;
            	}else {
            		mxGetPr(out0)[i]=(double)(data[i]);
            	}
            }

            plhs[0]=out0;
            break;

        case 2:
            out0=mxCreateDoubleMatrix(n_points/2,1,mxREAL);
            out1=mxCreateDoubleMatrix(n_points/2,1,mxREAL);

            for (i=0;i<n_points;i+=2)
            {
            	if ( i <= 44){
            		mxGetPr(out0)[i/2]=0.0;
            		mxGetPr(out1)[i/2]=0.0;
            	}else{
            		mxGetPr(out0)[i/2]=(double)(data[i]);
            		mxGetPr(out1)[i/2]=(double)(data[i+1]);
            	}
            }

            plhs[0]=out0;
            plhs[1]=out1;
            break;

        case 3:
            out0=mxCreateDoubleMatrix(n_points/3,1,mxREAL);
            out1=mxCreateDoubleMatrix(n_points/3,1,mxREAL);
            out2=mxCreateDoubleMatrix(n_points/3,1,mxREAL);

            for (i=0;i<n_points;i+=3)
            {
                mxGetPr(out0)[i/3]=(double)(data[i]);
                mxGetPr(out1)[i/3]=(double)(data[i+1]);
                mxGetPr(out2)[i/3]=(double)(data[i+2]);
            }

            plhs[0]=out0;
            plhs[1]=out1;
            plhs[2]=out2;
            break;

        default:
            createEmptyOutputMatrices(nlhs,plhs);
            return -1;
    }

    if (nlhs > numOutArgs)
    {
        for (i=numOutArgs;i<nlhs;i++)
        {
            plhs[i]=mxCreateDoubleMatrix(0, 0, mxREAL);
        }
    }
    return 0;
}




void mexFunction(int nlhs, mxArray *plhs[], int nrhs, const mxArray *prhs[]){

	static urg_t urg; //structure for urg control

	int data_max;
	long *data;
	int ret;
	int n;

	long com_baudrate;

    // Get input arguments
    if (nrhs == 0) {
        mexErrMsgTxt("hokuyoAPI: Need input argument");
    }

    const int BUFLEN = 256;
    char buf[BUFLEN];
    if (mxGetString(prhs[0], buf, BUFLEN) != 0) {
        mexErrMsgTxt("hokuyoAPI: Could not read string.");
    }

    if (strcasecmp(buf, "open") == 0){ //conncetion routine for urg device

        if (nrhs != 3){ 
            mexErrMsgTxt("hokuyoAPI: Please enter correct arguments: 'open', <device>, <baud rate>\n");
        }

        switch ((int)mxGetPr(prhs[2])[0]){
            case 19200:
                com_baudrate = 19200;
                break;
            case 115200:
                com_baudrate = 115200;
                break;
            default:
                mexErrMsgTxt("hokuyoAPI: Invalid hokuyo baud rate. Options are 19200, 115200. Using Default.");
                com_baudrate = 115200;
                break;
        }

        /* Connection */
        // If there are 3 args, then the 2nd should be the device 
        if (nrhs == 3){
        	if (mxGetString(prhs[1], buf, BUFLEN) != 0) {
        		mexErrMsgTxt("hokuyoAPI: Could not read string while reading the device name");
        	}

        	ret = urg_connect(&urg, buf, com_baudrate);

        	if (ret < 0) {
        		printf("hokuyoAPI: return urg_connect is < 0, exiting\n");
        		urg_exit(&urg, "urg_connect()");
        	}else{
        		printf("Connected with URG-04LX\n");
        	}
        }else{
        	mexErrMsgTxt("hokuyoAPI: Please enter correct arguments\n");
        }

    }else if(strcasecmp(buf, "getReading")== 0){ //get one reading
    	int getIntensity = 0;

    	if (nrhs == 2){
//    		printf("Intensity reading\n");
    		getIntensity = 1;
    	}else if (nrhs == 1){
//    		printf("normal reading\n");
    	}else{
    		printf("Wrong use. Use 'getReading', or 'getReading', 'intensity' for intensity readings as well\n");
    		return;
    	}

    	//if(urg_isConnected(&urg) == 0){
    		/* Reserve for reception data */
    		data_max = urg_dataMax(&urg);
    		data = (long*)malloc(sizeof(long) * data_max);
    		if (data == NULL) {
    			mexErrMsgTxt("hokuyoAPI: Malloc error!\n");
    		}

    		/* Request for GD data */
    		if(getIntensity){
    			ret = urg_requestData(&urg, URG_GD_INTENSITY, URG_FIRST, URG_LAST);
    		}else{
    			ret = urg_requestData(&urg, URG_GD, URG_FIRST, URG_LAST);
    		}
    		if (ret < 0) {
    			urg_exit(&urg, "urg_requestData()");
    		}

    		/* Reception */
    		n = urg_receiveData(&urg, data, data_max);
//    		printf("# n = %d\n", n);
    		if (n < 0) {
    			printf("Error receiving data\n");
    			urg_exit(&urg, "urg_receiveData()");
    		}else{
    			if (createOutput(&urg, data, urg_recentTimestamp(&urg), n, nlhs, plhs, getIntensity) == -1){
    				printf("Error creating output\n");
    			}
    		}

    		//urg_disconnect(&urg);
    		free(data);

    	//}else{
   // 		mexErrMsgTxt("URG device not connected. Connect using 'open'\n");

    	//}

    }else if(strcasecmp(buf, "close")== 0){
 //   	if(urg_isConnected(&urg)== 0){
    		urg_disconnect(&urg);
    		printf("Closing urg connection\n");
   // 	}else{
    //		mexErrMsgTxt("URG device not connected.\n");
    	//}

    }else if(strcasecmp(buf, "getParameters") == 0){ // opens device, get parameters and closes the connection

    	urg_parameter_t parameters;

        if (nrhs != 3){
             mexErrMsgTxt("hokuyoAPI: Please enter correct arguments: 'getParameters', <device>, <baud rate>\n");
         }

         switch ((int)mxGetPr(prhs[2])[0]){
             case 19200:
                 com_baudrate = 19200;
                 break;
             case 115200:
                 com_baudrate = 115200;
                 break;
             default:
                 mexErrMsgTxt("hokuyoAPI: Invalid hokuyo baud rate. Options are 19200, 115200. Using Default.");
                 com_baudrate = 115200;
                 break;
         }

         /* Connection */
         // If there are 3 args, then the 2nd should be the device
         if (nrhs == 3){
         	if (mxGetString(prhs[1], buf, BUFLEN) != 0) {
         		mexErrMsgTxt("hokuyoAPI: Could not read string while reading the device name");
         	}

         	ret = urg_connect(&urg, buf, com_baudrate);

         	if (ret < 0) {
         		printf("hokuyoAPI: return urg_connect is < 0, exiting\n");
         		urg_exit(&urg, "urg_connect()");
         	}else{
         		printf("Connected with URG-04LX\n");
         	}
         }else{
         	mexErrMsgTxt("hokuyoAPI: Please enter correct arguments\n");
         }

         /* Get sensor parameter */
         ret = urg_parameters(&urg, &parameters);
         printf("urg_getParameters: %s\n", urg_error(&urg));
         if (ret < 0) {
        	 printf("Could not get parameters \n");
        	 urg_disconnect(&urg);
         }

         urg_disconnect(&urg);

         /* Display */
         printf("distance_min: %ld\n", parameters.distance_min_);
         printf("distance_max: %ld\n", parameters.distance_max_);
         printf("area_total: %d\n", parameters.area_total_);
         printf("area_min: %d\n", parameters.area_min_);
         printf("area_max: %d\n", parameters.area_max_);
         printf("area_front: %d\n", parameters.area_front_);
         printf("scan_rpm: %d\n", parameters.scan_rpm_);
         printf("\n");

         /* Display information from URG structure (same resource as above) */
         printf("urg_getDistanceMax(): %ld\n", urg_maxDistance(&urg));
         printf("urg_getDistanceMin(): %ld\n", urg_minDistance(&urg));
         printf("urg_getScanMsec(): %d\n", urg_scanMsec(&urg));
         printf("urg_getDataMax(): %d\n", urg_dataMax(&urg));

    }else if(strcasecmp(buf, "continuousReading") == 0){ //using the MD/MS command

    	// Need an array of pointers to data arrays with scans
    	// Read the data in a for loop n times, put in n matrices.

    	int captureTimes = 1; //default 1
    	int i;

		int timestamp = -1;
    	int previous_timestamp;
    	int remain_times;
		int scan_msec;

    	if (nrhs == 2){ // assume that the capture times are the second argument.
    		captureTimes = (int)(mxGetPr(prhs[1])[0]);
    	}else {
    		printf("Need exactly 2 arguments. Usage: 'continuousReading', <no of captures>\n");
    		return;
    	}

    	urg_parameter_t parameter;


    	/* Reserve for receive buffer */
    	data_max = urg_dataMax(&urg);
    	data = (long*)malloc(sizeof(long) * data_max);
    	if (data == NULL) {
    		printf("Malloc error\n");
    		return;
    	}

    	urg_parameters(&urg, &parameter);
    	scan_msec = urg_scanMsec(&urg);

    	/* Request for MD data */
    	/* To get data continuously for more than 100 times, set capture times equal
    	     to infinity times(UrgInfinityTimes) */
    	/* urg_setCaptureTimes(&urg, UrgInfinityTimes); */

    	urg_setCaptureTimes(&urg, captureTimes);

    	/* Request for data */
    	ret = urg_requestData(&urg, URG_MD, URG_FIRST, URG_LAST);
    	if (ret < 0) {
    		urg_exit(&urg, "urg_requestData()");
    	}

    	for (i = 1; i < captureTimes; ++i) {
    		/* Reception */
    		n = urg_receiveData(&urg, data, data_max);
    		printf("n = %d\n", n);
    		if (n < 0) {
    			urg_exit(&urg, "urg_receiveData()");
    		} else if (n == 0) {
    			printf("n == 0\n");
    			--i;
    			continue;
    		}

    		/* Display the front data with timestamp */
    		/* Delay in reception of data at PC causes URG to discard the data which
    	       cannot be transmitted. This may  results in remain_times to become
    	       discontinuous */
    		previous_timestamp = timestamp;
    		timestamp = urg_recentTimestamp(&urg);
    		remain_times = urg_remainCaptureTimes(&urg);


    		if (remain_times <= 0) {
    			break;
    		}

    	}

    }else{
    	printf("doing nothing\n");
    }

    return;


}
\end{lstlisting}


\chapter{URG Communication Protocol}

%\fullpage[1-25]{urg-communication}

