\documentclass[a4paper, 10pt]{article}

\usepackage[utf8x, latin1]{inputenc}
\usepackage{amsmath,amssymb,amsthm}
\usepackage[english]{babel}
\usepackage{fullpage}
\usepackage{multirow}


\title{\begin{huge}\textbf{Project plan}\end{huge} \\ TTK4900: Master Thesis Project }
\author{Anders Garmo}

\begin{document}

\maketitle

\newpage

\section{Introduction}
The \emph{Master Thesis} at the Norwegian University of Science and Technology are the
last work on the masters degree in Engineering Cybernetics. This document will try to
predict the work flow of the project. 

\section{Product}
The product of this work will be a master thesis report. Test results from the fusion of
the sensor data will also be among the results. The project will also generate usable code
for the testing of the sensors. 

\section{Milestones and Goals}
Proper milestones and goals for the project are important to ensure a good work flow throughout the
project work. The goals are shown below. This is how I see how the project should evolve. 

\begin{enumerate}
    \item Literature study
        \begin{enumerate}
            \item Sensor study
            \item 3D/2D sensor fusion
            \item Representation of the sensor data
        \end{enumerate}
    \item Find possible solutions
        \begin{enumerate}
            \item Sketch integration of the sensors
            \item Find a sensor fusion approach
        \end{enumerate}
    \item Implementing the solutions and testing
        \begin{enumerate}
            \item Interface the sensors to control unit
            \item Implement a senor fusion algorithm
            \item Test the sensor fusion approach and gather data
        \end{enumerate}
    \item Discussion of the test results
        \begin{enumerate}
            \item Analyze the test results
            \item Identify possible error conditions
            \item Discuss performance of sensor fusion
        \end{enumerate}
    \item Final conclusions
\end{enumerate}


\section{Timeline}
Basically the master thesis work will be performed during 21 weeks of work in the spring
semester. The project start will be January 17th and the submit date is June 13th as stated in the
Master contract. 

\subsection{Testing}
During the implementation phase, tests need to be performed to see how the proposed
solution performs. This tests will likely be scheduled the last two weeks of April. There will be a number of
tests, which will be engineered to sort out the pro's and con's of the sensors. 


\section{Meetings}
There will be meetings with the supervisors once a week. In this meetings the work done in
the previous week will be talked about and the next weeks work will be discussed.  


\section{Milestones}
\begin{tabular}{| c | p{10.5cm} || c |}
    \hline
        Milestone   &   Activity    &   Deadline \\
    \hline
    \hline
        \multirow{6}{*}{\textbf{1}}  &   
                        \textbf{Literature Study} This should map out what is
                        stat-of-the-art now and what are the major research topics. & \\
                        & & \\
                        & $\bullet$ Study of 2D sensors & February 15 \\ 
                        & $\bullet$ Study of 3D sensors & February 15 \\ 
                        & $\bullet$ Evaluate pro's and con's of the sensors & February 15 \\ 
                        & $\bullet$ Study of Sensor Fusion techniques & February 22 \\
                        & $\bullet$ Application of the above in real applications & February 22 \\
        \hline
        \multirow{4}{*}{\textbf{2}}  &   
                        \textbf{Solution Modelling} Find a solution for the problem
                        at hand. This is the natural extension to the literature study,
                        and the two milestones will possibly fuse together. & \\
                        & & \\                        
                        & $\bullet$ Define the working space of the robot and sensors & Mars 8 \\
                        & $\bullet$ Find possible solutions to the problem & Mars 8 \\
                        & $\bullet$ Representation of the sensor data & Mars 15 \\
        \hline
        \multirow{8}{*}{\textbf{3}}  & 
                        \textbf{Implementation and Testing} This is probably the most
                        time consuming period. Testing of the scheme should be scheduled
                        around the last two weeks of April.  & \\
                        & & \\
                        & $\bullet$ Implement some interface with the 2D sensor & Mars 22  \\
                        & $\bullet$ Implement some interface with the 3D sensor & Mars 29 \\
                        & $\bullet$ Implement the sensor fusion strategy & April 12 \\
                        & $\bullet$ Implement some representation of the data & April 26 \\
                        & $\bullet$ Testing of the proposed scheme or recording the necessary data & Week 16 and 17 \\
                        & $\bullet$ Make changes and retest with recorded data, to get desired
                        results & May 24 \\
        \hline
        \multirow{6}{*}{\textbf{4}}  &   
                        \textbf{Discussion and Conclusion} This is the most important
                        aspect of the project, which justifies the time used on the
                        project. & \\
                        & &  \\
                        & $\bullet$ Discuss the test results & June 5 \\
                        & $\bullet$ Identify error conditions and why the results are as they are &
                                            June 5 \\
                        & $\bullet$ Discuss improvements and further work & June 13 \\
                        & $\bullet$ Finalize the report & June 13 \\
    \hline
\end{tabular}

\end{document}


